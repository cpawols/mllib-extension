\documentclass[magisterska]{pracamgr}
\usepackage[usenames,dvipsnames]{xcolor}

\usepackage[MeX,plmath]{polski}
\usepackage[utf8]{inputenc}
\usepackage{amsmath}
\usepackage{graphicx}
\usepackage{amsfonts}
\usepackage{amsthm}
\usepackage{enumerate}
\usepackage{color}
\usepackage{amsfonts}
\usepackage{fancyhdr}
\usepackage{amssymb}
\usepackage{amsthm}
\usepackage{textcomp}
\usepackage{tikz}

\usepackage{t1enc}
\usepackage{a4wide}
\usepackage{amsfonts}
\usepackage{amsmath}
\usepackage{enumerate}
\usepackage{verbatim}
\usepackage{geometry}
\geometry{left=1in,right=1in,%
bindingoffset=0mm, top=1in, bottom=1in}
\usepackage{amssymb, latexsym}
\usepackage{amsthm}
\usepackage{palatino}
\usepackage{array}
\usepackage[]{algorithm2e}
%\usepackage{pstricks}
\usepackage{enumitem}
\setlist[enumerate]{label*=\arabic*.}
\usepackage{textcomp}
\usepackage{listings}
\usepackage[export]{adjustbox}
\lstset{language=Python}

% Default fixed font does not support bold face
\DeclareFixedFont{\ttb}{T1}{txtt}{bx}{n}{12} % for bold
\DeclareFixedFont{\ttm}{T1}{txtt}{m}{n}{12}  % for normal

% Custom colors
\usepackage{color}
\definecolor{deepblue}{rgb}{0,0,0.5}
\definecolor{deepred}{rgb}{0.6,0,0}
\definecolor{deepgreen}{rgb}{0,0.5,0}
\definecolor{myred}{rgb}{0.7,0,0}

% Python style for highlighting
\newcommand\pythonstyle{\lstset{
language=Python,
basicstyle=\ttm,
commentstyle=\color{myred},
otherkeywords={self, map, reduce, reduceByKey, mapPartitions, collect, paralellize, yield},             % Add keywords here
keywordstyle=\ttb\color{deepblue},
emph={MyClass,__init__},          % Custom highlighting
emphstyle=\ttb\color{deepred},    % Custom highlighting style
stringstyle=\color{deepgreen},
frame=tb,                         % Any extra options here
showstringspaces=false            % 
}}

% Python environment
\lstnewenvironment{python}[1][]
{
\pythonstyle
\lstset{#1}
}
{}

% Python for external files
\newcommand\pythonexternal[2][]{{
\pythonstyle
\lstinputlisting[#1]{#2}}}

% Python for inline
\newcommand\pythoninline[1]{{\pythonstyle\lstinline!#1!}}

% Tu jest dobre miejsce na Twoje wżasne makra i~żrodowiska:
\newtheorem{thm}{Thm}[chapter]
\theoremstyle{plain}
\newtheorem{twierdzenie}[thm]{Twierdzenie}
\newtheorem{stwierdzenie}[thm]{Stwierdzenie}
\newtheorem{wniosek}[thm]{Wniosek}
\newtheorem{lemat}[thm]{Lemat}
\newtheorem{przyklad}[thm]{Przykład}

\theoremstyle{definition}
\newtheorem{definicja}[thm]{Definicja}

\theoremstyle{remark}
\newtheorem{uwaga}[thm]{Uwaga}
\newtheorem{zadanie}{Zadanie}

\renewcommand{\Re}{\mathrm{Re}}
\renewcommand{\Im}{\mathrm{Im}}
\newcommand{\wek}{\begin{pmatrix} k \\
		l \end{pmatrix}}

\newcommand{\wekx}{\begin{pmatrix} x \\
		y \end{pmatrix}}		

\newcolumntype{P}[1]{>{\centering\arraybackslash}p{#1}}

% koniec definicji


\author{Krzysztof Rutkowski}

\nralbumu{319379}

\title{Wybrane metody dyskretyzacji i generowania nowych cech z zastosowaniem paradygmatu MapReduce}

\tytulang{Discretization and feature extraction methods with application of MapReduce paradigm}

\kierunek{Matematyka}


\opiekun{prof.dr.hab Andrzej Skowron\\
  Zakład Logiki Matematycznej\\
  }

\date{Czerwiec 2016}

\dziedzina{ 
11.1 Matematyka\\ 
}

%Klasyfikacja tematyczna wedlug AMS (matematyka) lub ACM (informatyka)
\klasyfikacja{sztuczna inteligencja, systemy decyzyjne}

\keywords{discretization, feature extraction, decision system, mapreduce}

\begin{document}

%tu idzie streszczenie na strone poczatkowa
\begin{abstract}
TODO
\end{abstract}

\tableofcontents
%\listoffigures
%\listoftables


%\chapter{Wprowadzenie}
\chapter{Podstawowe informacje}\label{r:pojecia}

\begin{definicja}
Systemem decyzyjnym ($ang. \ Decision \ System$) nazwiemy następującą
trójkę obiektów $\mathbb{S} = (U, A \cup d)$, gdzie
\begin{itemize}
 \item U jest skończonym zbiorem, którego elementami są obiekty:
    \begin{align*}
	    U = \{u_1, \ldots, u_n \}  
    \end{align*}

 \item A jest skończonym zbiorem atrybutów warunkowych (cech) takim, że dla każdego $ a \in A$ istnieje funkja
    \begin{align*}
	  a:U \rightarrow V^{a}
    \end{align*}
    gdzie przez $V^{a}$ oznaczamy zbiór wartości atrybutów,
    
 \item d jest skończonym zbiorem atrybutów, które nazywamy atrybutami decyzyjnymi.
    \begin{align*}
     d: U \rightarrow \{d_1, \ldots d_k\},
    \end{align*}
  gdzie zbiór $\{d_1, \ldots d_k\}$, jest to zbiór wszystkich możliwych decyzji.
\end{itemize}

\end{definicja}

Jedną z możliwości reprezentacji systemu decyzyjnego  \ref{SystemDecyzyjny:1.1} jest dwuwymiarowa tabela.
W poszczególnych wierszach tabeli znajdują się obiekty $\{u_1, \ldots, u_n \}$,
natomiast w kolumnach znajdują się wartości atrybutów $\{a_{11}, \ldots, a_{nn} \}$.
Dokładniej mówiąc w $i$-tym wierszu znajduje się wartość  $j$-tego atrybutu ($a_{i}$) dla obiektu $u_{i}$.
W ostatniej kolumnie umieszczony jest atrybut decyzyjny $d_{i}$ dla obiektu $u_i$

\begin{center}
  \begin{tabular}{  l || l l l | r }
  \label{SystemDecyzyjny:1.1}
    $U$ & $a_1$ & $\ldots$ & $a_n$ & $dec$ \\ 
    \hline
  
    $u_1$ & $a_{11}$ & $\ldots $ & $a_{1n} $ & $d_{1}$ \\
    $u_2$ & $a_{21}$ & $\ldots $ & $a_{2n} $ & $d_{2}$ \\ 
    \ \ $\vdots$ & \ \ $\vdots$ & \ \ $\vdots$ & \   $\vdots$ & $\vdots$ \\ 
    $u_n$ & $a_{n1}$ & $\ldots $ & $a_{nn} $ & $d_{n}$ \\     
  \end{tabular}\par
  \bigskip 
  \ref{SystemDecyzyjny:1.1} : System decyzyjny

\end{center}

Obiektami tablicy decyzyjnej mogą być takie rzeczy jak szeregi czasowe, ludzie,
papiery wartościowe i inne. \newline Dla przykładu rozważmy tablicę decyzyjną,
której obiektami są ludzie. Rozważmy pacjentów, którzy zostali przebadani pod kątem białaczki.
Każdemu pacjentowi zbadano kilkanaście genów. Wynikiem tego padania jest kilkanaście 
liczb rzeczywistych opisujących poszczególne geny. Na podstawie wyników badań lekarz diagnozował pacjentów.
\newline
\begin{center}
 \begin{tabular}{l || P{2.5cm} P{2.5cm} P{2.5cm} | P{1.5cm}}
  $Pacjent$     & gen 1 & gen 2 & gen 3 & Chory \\ 
  \hline
  $pacjent_{1}$ & 0.02       & 0.12       & 1.2    & Tak \\
  $pacjent_{2}$ & -0.45      & 4.56       & 2.3    & Nie \\
  $pacjent_{3}$ & 1.32       & 2.3        & 0.01   & Tak \\
  $pacjent_{4}$ & 1.23       & 1.2        & 2.12   & Nie \\
 \end{tabular}
\end{center}
W powyższej sytuacji obiekt (pacjent) tablicy decyzyjnej zawiera ciąg liczb (każda traktowana jako
jeden atrybut), które zostały uzyskane podczas przeprowadzonego badnia. Ostatnim elementem wiersza
jest atrybut decyzyjny(decyzja), stwierdzająca czy pacjent jest chory.
\newline
Często zdarza się tak, że w tabeli występują wartości rzeczywiste, np w powyższych danych medycznych, gdzie 
wyniki badań mają wartości z pewnego przedziału bądź gdy mamy podane pewne ciągłe wielkości fizyczne, np
temperaturę. Z różnych względów, np wymagają tego niektóre klasyfikatory, chcieliśmy mieć tabelę z wartościami
nominalnymi. I tak, np poniższą tabelę:

\begin{center}
 \begin{tabular}{l || P{2.5cm} P{2.5cm} P{2.5cm} | P{1.5cm}}
  $U$     & Temparatura & Wilgotność & Zachmurzenie & Gra \\ 
  \hline
  $u_{1}$ & 0      & 30      & słabe          & Nie \\
  $u_{2}$ & 10     & 90      & umiarkowane    & Nie \\
  $u_{3}$ & 20     & 30      & słabe   	      & Tak \\
  $u_{4}$ & 15     & 30      & deszcz         & Nie \\
 \end{tabular}
\end{center}

Moglibyśmy zastąpić tabelą:

\begin{center}
 \begin{tabular}{l || P{2.5cm} P{2.5cm} P{2.5cm} | P{1.5cm}}
  $U$     & Temparatura & Wilgotność & Zachmurzenie & Gra \\ 
  \hline
  $u_{1}$ & niska      & mała      & słabe          & Nie \\
  $u_{2}$ & średnia    & duża      & umiarkowane    & Nie \\
  $u_{3}$ & wysoka     & mała      & słabe   	    & Tak \\
  $u_{4}$ & średnia    & mała      & deszcz         & Nie \\
 \end{tabular}
\end{center}

\newpage
Wprowadzę teraz wstępne pojęcia dotyczące dyskretyzacji atrybutów rzeczywistych.


\begin{definicja}[Niesprzeczna tabela decyzyjna]
Tabela  $\mathbb{S} =(U, A, \{d\} )$  jest niesprzeczna, jeśli dla dowolnych dwóch obiektów $u_{1}$ i $u_{2}$:
  \begin{align*}
     (a_{1}(u_{1}), a_{2}(u_{1}), ..., a_{k}(u_{1})) = (a_{1}(u_{2}), a_{2}(u_{2}), ..., a_{k}(u_{2})) \Rightarrow 
     d(u_{1}) = d(u_{2})
  \end{align*}.
Tj jeśli mamy te same wartości na atrybutach musimy mieć te same decyzje.
\end{definicja}

\begin{definicja}[Podział atrubutu]
Dla tabeli  $\mathbb{S} =(U, A, \{d\} )$  i zbioru cięć dla atrubutu $a$:
\begin{align*}
     c_{0}^{a} < c_{1}^{a} < ... < c_{k_{a}}^{a} < c_{k_{a} + 1}^{a}
  \end{align*}
definiujemy podział (dyskretyzacje) atrubutu $a$ jako:
  \begin{align*}
     P_{a} = {[c_{0}^{a}, c_{1}^{a}], ... , [c_{k_{a}}^{a}, c_{k_{a} + 1}^{a}]}
  \end{align*}
Zaś podział całego uniwersum jako:
  \begin{align*}
     P = \cup_{a \in A} P_{a}
  \end{align*}
\end{definicja}

\begin{uwaga}
Szukanie niesprzecznego zbioru cięć o najmniejszej mocy jest $NP$-zupełny.
Dowód poprzez sprowadzenie do problemu $SET COVER$.
\end{uwaga}


\chapter{Skalowalność}

Większość algorytmów została zaimplementowana w sposób skalowalny za pomocą frameworku 
do obliczeń równoległych $Apache Spark$ bazującego na paradygmacie MapReduce.
Na wstępie przytoczę kilka definicji przydatnych w dalszej części omawiania Sparka.

\section{Wstępne definicje}
\begin{definicja}[Programowanie funkcyjne]
Filozofia i metodyka programowania będąca odmianą programowania deklaratywnego, 
w której funkcje należą do wartości podstawowych, a nacisk kładzie się na wartościowanie 
(często rekurencyjnych) funkcji, a nie na wykonywanie poleceń.
W czystym programowaniu funkcyjnym, raz zdefiniowana funkcja zwraca zawsze 
tę samą wartość dla danych wartości argumentów, tak jak funkcje matematyczne.
W czystych językach funkcyjnych nie występują zmienne ani efekty uboczne, 
a wartościowanie jest leniwe, tj wykonywanie obliczenia następuje tylko w momencie, gdy jest to potrzebne.
\end{definicja}

\begin{definicja}[Klaster]
Grupa połączonych jednostek komputerowych, które współpracują ze sobą w celu udostępnienia zintegrowanego środowiska pracy.
Komputery wchodzące w skład klastra (będące członkami klastra) nazywamy węzłami (ang. node).
\end{definicja}

\begin{definicja}[HDFS]
Hadoop Distributed File System, rozproszony, skalowalny, spójny system plików używany przez framework Apache Spark.
\end{definicja}

\begin{definicja}[RDD]
Resilent Distributed Dataset, podstawowa struktura danych w Sparku. RDD jest kolekcją elementów dystrybuowaną pomiędzy 
węzły klastra, na których operacje mogą być wykonywane równolegle. Po zakończenie operacji na RDD można je dalej 
przechowywać w pamięci RAM, co umożliwia szybsze dostęp do następnych obliczeń. 
\end{definicja}


\section{Paradygmat MapReduce}

Paradygmat MapReduce został wprowadzony przez firmę $Google$ i służy do równoległego przetwarzania
dużych danych przy użyciu klastra.
Rozproszenie obliczeń pomaga w istotny sposób zmniejszyć czas wykonywanych zadań.
Apache Spark jest niejako rozszerzeniem wprowadzonego paradygmatu. 
Jak każdy framework do obliczeń na klastrze posiada pod sobą rozproszony system plików (HDFS), 
a także oprogramowanie (dla workerów i mastera), które odpowiada za zarządzanie awariami, przydzielanie 
zadań i organizację. 
\newpage

\begin{figure}
 \caption{architektura Sparka}
 \includegraphics[scale=0.8]{spark-architecture.png}
\end{figure}

Spark jest dedykowany dla języka funkcyjnego - Scali, algorytmy zostały jednak napisane w pythonie, który 
mimo, iż językiem funkcyjnym nie jest, posiada jednak cechy języków funkcyjnych. 

Typowy potok obliczeń w Sparku polega na:
\begin{itemize}
 \item Stworzeniu RDD bazującego na kolekcji utrzymywanej w programie
 \item Wykonaniu transformacji na RDD
 \item Wykonaniu akcji
\end{itemize}

Funkcje w Sparku dzielą się na transformacje i akcje. 
\begin{itemize}
 \item \textbf{Transformacja} jest funkcją działającą na RDD i zwracającą RDD
 \item \textbf{Akcja} jest funkcją działającą na RDD i zwracającą kolekcję do programu po zakończeniu obliczeń
\end{itemize}
Zarówno akcje, jak i transformacje wykonywane są leniwie. Przykłady transformacji:
\begin{itemize}
 \item \textbf{map(f)} - zwraca nowy RDD aplikując funkcję $f$ do każdego elementu RDD
 \item \textbf{filter(f)} - zwraca nowy RDD składający się tylko z tych elementów, dla których funkcja $f$ zwróci True
 \item \textbf{reduceByKey(f)} - dla RDD postaci (klucz, wartość), grupuje według kluczy i zwraca RDD z agregacją wartości 
 za pomocą funkcji $f$ dla zgrupowanych kluczy
\end{itemize}

Przykłady akcji:
\begin{itemize}
 \item \textbf{reduce(f)} - zwraca kolekcję będą agregacją RDD za pomocą funkcji $f$, która przyjmuje dwa 
 argumenty i zwraca jeden
 \item \textbf{collect()} - zwraca kolekcję elementów do programu
\end{itemize}

W powyższych przykładach zamiast funkcji $f$ można użyć wyrażeń lambda (funkcji anonimowych).

Przykład, zliczanie słów w języku python:
\begin{python}
#ustawiamy kontekst i konfiguracje
conf = (SparkConf().setMaster("spark://localhost:7077").
		    setAppName("entropy"))
sc = SparkContext(conf=conf)
words = ["ala", "ma", "ala", "kota", "ma"]
#tworzenie rdd na podstawie listy
words_rdd = sc.paralellize(words)  
words_counts =  words_rdd.
		map(lambda x: (x, 1)).
		reduceByKey(lambda x, y: x+y)
\end{python}

W powyższym, jak i w następnych listingach algorytmów w języku python, sc zawsze będzie obiektem klasy $SparkContext$, który inicjuje
RDD na podstawie konfiguracji klastrowego środowiska dla Sparka. Konfigurację tworzymy na podstawie obiektu klasy $SparkConf$, bądź
ustawiamy parametry klastra, takie jak:
\begin{itemize}
 \item liczba komputerów
 \item liczba procesorów
 \item pamięć RAM na jednostkę obliczeniową
 \item nazwa aplikacji
\end{itemize}
w odpowiednim pliku konfiguracyjnym. 


Obliczenia wykonywane w Sparku są nie tylko skalowalne, ale również szybkie ze względu na:
\begin{itemize}
 \item leniwość wykonywanych operacji, obliczenia wykonywane są tylko wtedy gdy są potrzebne (ważna cecha
 języków funkcyjnych)
 \item możliwość przechowywania rozproszonych struktur danych (RDD) w pamięci RAM, co powoduje szybszy dostęp do danych
 podczas ponownego przetwarzania (np w Hadoop MapReduce musieliśmy dane wczytywać z HDFS)
\end{itemize}

\newpage

\begin{figure}
 \caption{porównanie szybkości działania Hadoop MapReduce i Spark na przykładzie regresji logistycznej}
 \includegraphics[scale=0.8]{spark-hadoop.png}
\end{figure}


\chapter{Skalowalne algorytmy dyskretyzacji}

W tym rozdziale omówię zaimplementowane algorytmy dyskretyzacji, jak i przedstawię wyniki eksperymentów i obliczeń.
Algorytmy dyskretyzacji dzielimy na:
\begin{itemize}
 \item \textbf{globalne} - szukają cięć na podstawie całej tabeli
 \item \textbf{lokalne} - szukają cięć na podstawie każdego atrubutu niezależnie
\end{itemize}
Będziemy tu analizować algorytmy lokalne, gdyż
w naturalny sposób łatwo je zrównoleglić, zgodnie ze schematem:
\begin{python}
# tworzymy RDD na podstawie listy kolumn i 
# dzielimy na $numer_of_columns$ kawalkow
columns_rdd = sc.paralellize(table, numer_of_columns)
# wykonujemy algorytm dyskretyzacyjny rownolegle dla kazdej kolumny
discretized_columns = columns_rdd.
		      mapPartitions(discretizeColumn).
		      collect()
\end{python}
Gdzie funkcja discretizeColumn przyjmuje kolumnę liczb rzeczywistych tabeli decyzyjnej i zwraca
jej dyskretyzację.
Metody globalne szukają zbioru cięć dla całej tabeli, przez co trudniej je zrównoleglić.
Niżej przedstawię 3 lokalne algorytmy dyskretyzacji.

\section{Sprawdzanie spójności tabeli}
By dyskretyzacja miała sens, potrzebujemy sprawdzić czy tabela jest spójna, w Sparku możemy 
to w prosty sposób zrobić za pomocą poniższego algorytmu:
\begin{python}
# funkcja agregujaca, zwraca tuple skladajaca sie z roznych decyzji
# dla danej kombinacji atrybutow
def get_unique_dec(x, y):
  return tuple(set(x + y))
# rows_rdd = (row, (dec,))
rows_rdd = sc.paralellize(table, numer_of_rows)
one_dec = rows_rdd.
	  reduceByKey(get_unique_dec).
	  map(lambda k, v: len(v) == 1).
	  collect()
# tabela jest spojna jesli dla kazdej kombinacji atrybutow
# mamy jedna decyzje
is_consistent = all(one_dec)
\end{python}
\section{Opis algorytmów}
Wszystkie z poniższych algorytmów są zrównoleglane ze względu na kolumnę, tak 
jak było to napisane we wstępie. Różnią się metodą $discretizeColumn$.
\subsection{Sortowanie}
Wszystkie niżej wymienione algorytmy operują na posortowanych kolumnach. By posortować całą tabelę najpierw
sprowadzam ją do postaci EAV, tj listy trójek (Obiekt, Atrybut, Wartość), a następnie sortuję tę listę 
najpierw względem wartości, potem obiektów, a następnie atrybutów, otrzymująć listę posortowanych wartości
zgrupowanych względem atrybutów. W Sparku wykonujemy sortowanie równoległe, za pomocą poniższego algorytmu,
będącego uogólnieniem mergesort:
\begin{itemize}
 \item dzielę listę na $numchunks$ kawałków
 \item sortuję każdy kawałek niezależnie
 \item scalam posortowane kawałki
\end{itemize}
Kod algorytmu w języku python:
\begin{python}
# funkcja porownujaca
# wzgledem atrybutu, potem wartosci, a nastepnie nazwy obiektu
def compare(iterator):
  yield sorted(iterator, key=lambda x: (x[1], x[2], x[0]))

def merge_sort():
  numChunks = 10
  eav_rdd_part = Configuration.sc.parallelize(eav, numChunks)
  self.eav =  eav_rdd_part.
	      mapPartitions(compare).
	      reduce(lambda x, y: 
	      sorted(x+y, key=lambda x: (x[1], x[2], x[0])))
\end{python}

\subsection{Algorytm bazowy}
Jest to prosty algorytm polegający na wykonywania cięć w równych odstępach, co $m$ obiektów, gdzie
$m$ jest parametrem. Przy czym kolumna 
Jego zaletą jest prostota, jednak wogóle nie patrzymy na decyzję.
Kod algorytmu:
\begin{python} 
def discretizeColumn(column, m):
  for ind, row in enumerate(column):
    yield (row[0], row[1], ind / m)
\end{python}

\begin{przyklad}[Przykładowa dyskretyzacja]
Dla tabeli:
\begin{center}
 \begin{tabular}{l || P{2.5cm} P{2.5cm} P{2.5cm} | P{1.5cm}}
  $U$     & Temperatura & Wilgotność & Zachmurzenie & Gra \\ 
  \hline
  $u_{1}$ & 0      & 30      & słabe          & Nie \\
  $u_{2}$ & 10     & 90      & umiarkowane    & Nie \\
  $u_{3}$ & 20     & 30      & słabe   	      & Tak \\
  $u_{4}$ & 15     & 50      & deszcz         & Nie \\
  $u_{5}$ & 15     & 30      & deszcz         & Nie \\
  $u_{6}$ & 30     & 70      & deszcz         & Nie \\
  $u_{7}$ & 15     & 30      & deszcz         & Nie \\
 \end{tabular}
\end{center}
Oraz $m$ = 3, mamy następującą dyskretyzację (dla kolumn Temperatura i Wilgotność):
\begin{center}
 \begin{tabular}{l || P{2.5cm} P{2.5cm} P{2.5cm} | P{1.5cm}}
  $U$     & Temperatura & Wilgotność & Zachmurzenie & Gra \\ 
  \hline
  $u_{1}$ & 0     & 0      & słabe          & Nie \\
  $u_{2}$ & 0     & 2      & umiarkowane    & Nie \\
  $u_{3}$ & 1     & 0      & słabe   	    & Tak \\
  $u_{4}$ & 0     & 1      & deszcz         & Nie \\
  $u_{5}$ & 1     & 0      & deszcz         & Nie \\
  $u_{6}$ & 2     & 1      & deszcz         & Nie \\
  $u_{7}$ & 1     & 1      & deszcz         & Nie \\
 \end{tabular}
\end{center}
\end{przyklad}
\subsection{OneR Discretizer}
W tym algorytmie wykonujemy cięcie w momencie, gdy
jednej z decyzji jest więcej niż innych, przy czym dany przedział musi zawierać więcej niżej
$m$ elementów, gdzie $m$ jest ustalonym parametrem.
Algorytm:
\begin{python}
def discretize_column(column, m):
        new_dec = 0
        block_elements = 0
        dec_histogram_block = Counter()
        for elem in column:

            if block_elements > m and
	       most_common(dec_histogram_block) > block_elements / 2:
                block_elements = 0
                new_dec += 1
                dec_histogram_block = Counter()
            dec_histogram_block[decision(elem)] += 1
            block_elements += 1

            yield (elem[0], elem[1], new_dec)
\end{python}
\begin{przyklad}[Przykładowa dyskretyzacja]
Dla tabeli:
\begin{center}
 \begin{tabular}{l || P{2.5cm} P{2.5cm} P{2.5cm} | P{1.5cm}}
  $U$     & Temperatura & Wilgotność & Zachmurzenie & Gra \\ 
  \hline
  $u_{1}$ & 0      & 30      & słabe          & Nie \\
  $u_{2}$ & 10     & 90      & umiarkowane    & Nie \\
  $u_{3}$ & 20     & 30      & słabe   	      & Tak \\
  $u_{4}$ & 15     & 50      & deszcz         & Nie \\
  $u_{5}$ & 15     & 30      & deszcz         & Tak \\
  $u_{6}$ & 30     & 70      & deszcz         & Nie \\
  $u_{7}$ & 15     & 30      & deszcz         & Nie \\
 \end{tabular}
\end{center}
Oraz $m$ = 4, mamy następującą dyskretyzację (dla kolumn Temperatura i Wilgotność):
\begin{center}
 \begin{tabular}{l || P{2.5cm} P{2.5cm} P{2.5cm} | P{1.5cm}}
  $U$     & Temperatura & Wilgotność & Zachmurzenie & Gra \\ 
  \hline
  $u_{1}$ & 0     & 0      & słabe          & Nie \\
  $u_{2}$ & 0     & 1      & umiarkowane    & Nie \\
  $u_{3}$ & 1     & 0      & słabe   	    & Tak \\
  $u_{4}$ & 0     & 0      & deszcz         & Nie \\
  $u_{5}$ & 0     & 0      & deszcz         & Tak \\
  $u_{6}$ & 1     & 1      & deszcz         & Nie \\
  $u_{7}$ & 1     & 0      & deszcz         & Nie \\
 \end{tabular}
\end{center}
\end{przyklad}
\subsection{Dyskretyzacja za pomocą entropii}
Entropię dla zbioru obiektów $U$ definiujemy jako:
\begin{align*}
  Ent(U) = - \sum_{d \in dec(U)} p_d \cdot log(p_d)
\end{align*}
Gdzie $dec(U)$ to zbiór decyzji obiektów ze zbioru $U$, oraz $p_d$ jest 
empirycznym prawdopodobieństwem decyzji $d$, liczonym z wzoru:
\begin{align*}
  p_d = \frac{|\{u \in U: dec(u) = d\}|}{|U|}
\end{align*}
Entropia jest miarą informacji jaką mamy o danym zbiorze obiektów.
Jest ona zerowa, gdy w danym zbiorze obiektów mamy jedną decyzję, a więc 
mamy pełną informację, zaś największa gdy wszystkich decyzji jest po równo. 
Entropię dla danego cięcia $c$ na danym atrybucie $a$ możemy zdefiniować jako:
\begin{align*}
  E(a, c, U) = \frac{U_1}{n}Ent(U_1) + \frac{U_2}{n}Ent(U_2)
\end{align*}
Gdzie $n$ jest liczbą obiektów w $u$ oraz $U_1$ i $U_2$ powstają z $U$ na podstawie cięcia $c$.
Dla danego cięcia możemy więc zdefniować miarę przyrostu informacji:
\begin{align*}
 Gain(a, c, U) = Ent(U) - E(a, c, U)
\end{align*}
Wybieramy takie cięcie, by zmaksymalizować zysk informacji. Cięcia wykonujemy rekurencyjnie,
dopóki nie spełniony zostanie pewien warunek stopu, tj gdy zysk informacji, który osiągamy
dla danego optymalnego cięcia jest mniejszy niż jakaś wielkość.
\subsubsection{Szukanie optymalnego cięcia - algorytm liniowy}
Możemy łatwo znależć optymalne cięcie dla danego zbioru obiektów w czasie $O(n^2)$ gdzie $n$ jest
liczbą obiektów za pomocą algorytmu (gdzie Ent(U) jest funkcją obliczającą entropię obiektów ze zbioru U):
\begin{python}
min_cut = 1
# mozliwe ciecia - pomiedzy obiektami od 1 do n
possible_cuts = range(1, n)
# ustawiamy wartosc minimalnego ciecia na nieskonczonosc
min_cut_value = INFINITY
for cut in possible_cuts:
  cut_value = cut / n * Ent(U[cut:]) + 
	      (n - cut) / n * Ent(U[:cut])
  if cut_value < min_cut_value:
    min_cut_value = cut_value
    min_cut = cut	      
\end{python}
Możemy jednak nie obliczać za każdym razem entropii, lecz liczyć ją na podstawie 
wcześniejszych wyników w czasie stałym, otrzymując algorytm liniowy ze względu na liczbę 
obiektów. Algorytm opiera się na obserwacji (zakładamy, że dla obiektów $1, ..., k-1$ mamy już obliczone 
częstości poszczególnych decyzji (słownik $decHist$ w którym trzymamy histogram decyzji, za pomocą którego w czasie stałym
liczymy $p_d$):
\begin{align*}
  Ent(U_k) = Ent(U_{k-1}) + log(p_{dec(u_k)}) \cdot p_{dec(u_k)} \\
  decHist[dec(u_k)] += 1 \\
  Ent(U_k) = Ent(U_{k-1}) - log(p_{dec(u_k)}) \cdot p_{dec(u_k)}
\end{align*}
Analogicznie trzymamy histogram dla drugiej obiektów $k, ..., n$ aktualizując entropię drugiej połowy.
Otrzymujemy więc następujący algorytm liniowy:
\begin{python}
def find_cut(U):
  # decHist1 -> histogram dla 1 połowy
  # decHist2 -> histogram dla 2 połowy
  # p1 -> prawdopodobieństwo liczone na podstawie decHist1
  # p2 -> prawdopodobieństwo liczone na podstawie decHist2
  def updateEnt(U, k):
    Ent[U[k:]] = Ent([[k-1:]] - (-log(p1[dec(u_k)])*p1[dec(u_k)])
    Ent[U[:k]] = Ent[U[:k+1]] - (-log(p2[dec(u_k)])*p2[dec(u_k)])
    decHist1[dec(U[k])] += 1    
    decHist2[dec(U[k])] -= 1
    Ent[U[k:]] = Ent[U[k-1:]] - (-log(p1[dec(u_k)])*p1[dec(u_k)])
    Ent[U[:k]] = Ent[U[:k+1]] - (-log(p2[dec(u_k)])*p2[dec(u_k)])
    return Ent[U[k:]], Ent[U[:k]]
    
  min_cut = 1
  # mozliwe ciecia - pomiedzy obiektami od 1 do n
  possible_cuts = range(1, n)
  # ustawiamy wartosc minimalnego ciecia na nieskonczonosc
  min_cut_value = INFINITY
  Ent[0] = 0
  for cut in possible_cuts:
    Ent1, Ent2 = updateEnt(U, cut)
    cut_value = cut / n * Ent1 + 
		(n - cut) / n * Ent2
    if cut_value < min_cut_value:
      min_cut_value = cut_value
      min_cut = cut	
  # jesli nie jest spelnione kryterium stopu zwracamy minimalne ciecie
  # oraz ciecia dla odpowiednich podzbiorow obiektow
  if not stop_criterion_satisfied(U, min_cut_value):
    yield min_cut
    
    for cut in find_cut(U[k:]):
      yield cut 
    for cut in find_cut(U[:k]):
      yield cut
\end{python}
Koncowy Algorytm:
\begin{python}
# zwracamy tabele z odpowiednimi wartosciami disc_value
def discretize_column(column):
  disc_value = 0
  cuts_set = sorted(find_cut_lin(column))
  cur_cut = cuts_set[0]

  for i, elem in enumerate(column):
    if cur_cut == i:
      disc_value += 1
    if len(cuts_set) > disc_value:
      cur_cut = cuts_set[disc_value]

    yield (elem[0], elem[1], dis_value)
\end{python}
\begin{przyklad}[Przykładowa dyskretyzacja]
Dla tabeli:
\begin{center}
 \begin{tabular}{l || P{2.5cm} P{2.5cm} P{2.5cm} | P{1.5cm}}
  $U$     & Temperatura & Wilgotność & Zachmurzenie & Gra \\ 
  \hline
  $u_{1}$ & 0      & 30      & słabe          & Tak \\
  $u_{2}$ & 10     & 90      & umiarkowane    & Nie \\
  $u_{3}$ & 20     & 30      & słabe   	      & Nie \\
  $u_{4}$ & 15     & 50      & deszcz         & Tak \\
  $u_{5}$ & 15     & 30      & deszcz         & Nie \\
  $u_{6}$ & 30     & 70      & deszcz         & Nie \\
  $u_{7}$ & 15     & 30      & deszcz         & Nie \\
 \end{tabular}
\end{center}
mamy następującą dyskretyzację (dla kolumn Temperatura i Wilgotność):
\begin{center}
 \begin{tabular}{l || P{2.5cm} P{2.5cm} P{2.5cm} | P{1.5cm}}
  $U$     & Temperatura & Wilgotność & Zachmurzenie & Gra \\ 
  \hline
  $u_{1}$ & 0     & 0      & słabe          & Tak \\
  $u_{2}$ & 0     & 2      & umiarkowane    & Nie \\
  $u_{3}$ & 1     & 1      & słabe   	    & Nie \\
  $u_{4}$ & 0     & 2      & deszcz         & Tak \\
  $u_{5}$ & 1     & 1      & deszcz         & Nie \\
  $u_{6}$ & 1     & 2      & deszcz         & Nie \\
  $u_{7}$ & 1     & 1      & deszcz         & Nie \\
 \end{tabular}
\end{center}
\end{przyklad}
\section{Wyniki eksperymentów}


\chapter{Generowanie nowych cech na podstawie algorytmów genetycznych}

W tym rozdziale omówię uogólnienie pojęcia dyskretyzacji jako generowanie nowych cech i omówię
znajdowanie nowych cech za pomocą hiperpłaszczyzn i algorytmu genetycznego.

\section{Dyskretyzacja jako generowanie nowych cech oraz generowanie nowych cech przez hiperpłaszczyzny}
Na dyskretyzację, a konkretnie na cięcie na pewnym atrybucie rzeczywistym możemy też patrzeć
jak na nową cechę zdefiniowaną przez funkcję charakterystyczną odpowiedniej półprzestrzeni. Na cięcie cięcia o wartości
$c$ na atrybucie $a_k$ możemy patrzeć jak na cechę:
\begin{align*}
v(a_1(u), a_2(u), ..., a_n(u)) = \begin{cases} 1 \text{ dla } a_k(u) \leq c \\ 0 \text{ dla } a_k(u) > c \end{cases}
\end{align*}
Dyskretyzacja na danym atrybucie definiuje nam jedynie hiperpłaszczyzny równoległe do odpowiednich osi. 
Możemy zdefiniować nowe cechy jako funkcje charakterystyczne ogólnych półprzestrzeni:
\begin{align*}
v(a_1(u), a_2(u), ..., a_n(u)) = \begin{cases} 1 \text{ dla } w_1 \cdot a_1(u) + w_2 \cdot a_2(u) + ... + w_n \cdot a_n(u) \leq c 
\\ 0 \text{ dla } w_1 \cdot a_1(u) + w_2 \cdot a_2(u) + ... + w_n \cdot a_n(u) > c \end{cases}
\end{align*}
Ogólne półprzestrzenie mają o wiele większą moc wyrazu i zwykle wystarczy ich mniej by znależć minimalny zbiór cięć dla
danej przestrzeni, tj tak podzielić przestrzeń, aby otrzymać spójną tabelę.
Jednak przestrzeń poszukiwań odpowiednich parametrów, tj wag $w_k$ jest nieskończona. Dla danego zbioru wag, tj dla danej półprzestrzeni,
możemy jednak ocenić jak dobre jest dane cięcie na podstawie funkcji oceny. Jedną z popularniejszych miar jest miara rozróżnialności, 
zdefiniujmy $C_i^R$ oraz $C_i^L$ jako liczbę obiektów o decyzji $i$ należących do odpowiednich półprzestrzeni:
\begin{align*}
C_i^R(H) = |\{u \in U: dec(u) = i \wedge w_1 \cdot a_1(u) + w_2 \cdot a_2(u) + ... + w_n \cdot a_n(u) > c\}| \\
C_i^L(H) = |\{u \in U: dec(u) = i \wedge w_1 \cdot a_1(u) + w_2 \cdot a_2(u) + ... + w_n \cdot a_n(u) \leq c\}|
\end{align*}
Na podstawie tych wielkości definiujemy miarę rozróżnialności dla hiperpłaszczyzny $H$:
\begin{align*}
discMeasure(H) = \sum_{i \neq j} C_i^R(h) \cdot C_j^L(H)
\end{align*}
Mając funkcję oceny możemy więc z powodzeniem stosować pewne heurystyki doboru wag i algorytmy genetyczne. Konkretny algorytm omówię w następnym
rozdziale.
\section{Algorytmy ewolucyjne}
Algorytmy ewolucyjne są jedną popularniejszych metod heurystycznych metod optymalizacji, jak np symulowane wyżarzanie. Sprawdzają się,
gdy mamy dobrze określoną funkcję celu oraz potrafimy odpowiednio zamodelować nasze uniwersum. Algorytm genetyczny polega na iteracyjnym 
poprawianiu wyniku poprzez dobieranie do następnej fazy tych osobników, które osiągają najlepsze wartośc funkcji celu. Podstawowym
pojęciem jest tu populacji i osobnik. Osobnik jest pewnym obiektem matematycznym, najczęściej wektorem zerojedynkowym, który modeluje
wielkości, które chcemy optymalizować. W przypadku problemu szukania optymalnej hiperpłaszczyzny będzie to wektor reprezentujący wagi $w_i$.
Populacja zaś jest zbiorem osobników. Algorytm genetyczny można podzielić na następujące etapy:
\begin{enumerate}
 \item Inicjowanie populacji początkowej
 \begin{enumerate}
  \item Inicjowanie losowe
  \item Inicjowanie wg wcześniej ustalonego schematu
 \end{enumerate}
 \item W pętli, aż do spełnienia pewnego warunku stopu wykonuj:
 \begin{enumerate}
  \item Liczymy funkcję oceny każdego osobnika w populacji. Najlepsze osobniki poddajemy reprodukcji.
  \item Proces repropdukcji najlepszych osobników:
  \begin{enumerate}
   \item Krzyżowanie - łączenie genotypów rodziców
   \item Mutacja - wprowadzanie losowych zmian
  \end{enumerate}
  \item Po procesie reprodukcji wybieramy najlepsze osobniki.
  \item Jeśli został spełniony warunek stopu wybieramy i zwracamy najlepszego osobnika w populacji.
 \end{enumerate}
\end{enumerate}
Podstawowymi zadaniami przy projektowaniu algorytmu genetycznego są:
\begin{itemize}
 \item ustalenie genomu osobnika jako reprezentanta wyniku
 \item ustalenie funkcji oceny
 \item dobór parametrów (warunek stopu, prawdopodobieństwo krzyżowania, mutacji, rozmiar populacji)
\end{itemize}
Dla problemu znajdowania optymalnej hiperpłaszczyzny funkcją oceny będzie wspomniana wcześniej miara $discMeasure$.
Pozostaje ustalić genom osobnika (hiperpłaszczyzny). Będziemy reprezentować hiperpłaszczyznę w przestrzeni $\mathbb{R}^n$
jako zbiór $n-1$ liniowo niezależnych wektorów i przesunięcia. 
\subsection{Reprezentacja hiperpłaszczyzny}
Ustalmy jedną z osi, niech to będzie $x_1$ odpowiadająca atrybutowi $a_1$ i liczbę naturalną $b$, będzie to parametr odpowiadający
za zbiór, z którego inicjujemy populację początkową. Z każdej dwu-wymiarowej płaszczyzny $L(x_1, x_i)$ wybieramy wektory 
$v_1^i, v_2^i, ..., v_{2^b}^i$ (nierównoległe do $x_1$) zdefiniowane przez:
\begin{align*}
v_j^i = [\alpha_j^i, 0, ..., 0, \underbrace{1}_{\text{i-ta pozycja}} , 0, ..., 0] \text{ dla } i=2, ..., n \text{ oraz } j=1, ..., 2^b 
\end{align*}
Wektory te będziemy inicjować przez losowy dobór $2^b$ liczb: $\alpha_1^i, \alpha_2^i, ..., \alpha_{2^b}^i$.
Każdy z wektorów $v_j^i$ może być reprezentowany przez $b$ bitów odpowiadających reprezentacji
binarnej liczby $j$. Rozważmy zbiór $n-1$ wektorów $\{v_{j_2}^2, v_{j_3}^3, ..., v_{j_n}^n\}$:
\begin{align*}
 v_{j_2}^2 = [\alpha_{j_2}^2, 1, 0, 0,  ..., 0] \\
 v_{j_3}^3 = [\alpha_{j_3}^3, 0, 1, 0,  ..., 0] \\
 ..... \\
 v_{j_n}^n = [\alpha_{j_n}^n, 0, 0, 0,  ..., 1]
\end{align*}
Łatwo widać, że wektory dla dowolnego doboru indeksów $j_2, j_3, ..., j_n$ wektory $\{v_{j_2}^2, v_{j_3}^3, ..., v_{j_n}^n\}$ 
są liniowo niezależne. Wybierając przesunięcie $P_1 = (p, 0, ..., 0)$ na osi $x_1$, możemy zdefiniować 
hiperpłaszczyznę $(P_1, v_{j_2}^2, v_{j_3}^3, ..., v_{j_n}^n)$ przez:
\begin{align*}
 H = \{(x_1, x_2, ..., x_n\} \in \mathbb{R}^n : x_1 - \alpha_{j_2}^2 x_2 - \alpha_{j_3}^3 x_3 - ... - \alpha_{j_n}^n x_n - p = 0\}
\end{align*}
Osobniki będziemy reprezentować przez $n-1$ liczb b-bitowych $\alpha_1, \alpha_2, ..., \alpha_{n-1}$, które będą 
reprezentować wektory (dla osi $x_1$):
\begin{align*}
 v_1 = [\alpha_1, 1, 0, 0,  ..., 0] \\
 v_2 = [\alpha_2, 0, 1, 0,  ..., 0] \\
 ..... \\
 v_{n-1} = [\alpha_{n-1}, 0, 0, 0,  ..., 1]
\end{align*}
Są one liniowo niezależne i rozpinają $n-1$ wymiarową podprzestrzeń liniową.
Taka reprezentacja umożliwia nam sprawdzenie w czasie $O(n)$ po której stronie hiperpłaszczyzny znajduje się dany obiekt $u \in U$ oraz 
policzyć w czasie $O(n)$ rzut wektora $u$ na $x_1$ równoległy do $L = lin(v_{j_2}^2, v_{j_3}^3, ..., v_{j_n}^n)$. Odpowiednie funkcje, $Test(u)$
licząca, po której stronie hiperpłaszczyzny jest obiekt $u$:
\begin{align*}
 Test(u) = \begin{cases} 1 \text{ jeśli } a_1(u) - \alpha_{j_2}^2 a_2(u) - \alpha_{j_3}^3 a_3(u) - 
 ... - \alpha_{j_n}^n a_n(u) \geq p \\ 0 \text{ wpp } \end{cases}
\end{align*}
Oraz $Projection(u)$ licząca rzut obiektu $u$ na odpowiednią oś (tu $x_1$):
\begin{align*}
 Projection(u) = a_1(u) - \alpha_{j_2}^2 a_2(u) - \alpha_{j_3}^3 a_3(u) - ... - \alpha_{j_n}^n a_n(u)
\end{align*}
\subsection{Szukanie optymalnej hiperpłaszczyzny za pomocą algorytmu genetycznego}
Za pomocą algorytmu genetycznego będziemy chcieli szukać optymalnej hiperłpaszczyzny dla danej osi. Dla każdego osobnika liczymy rzut obiektów na 
odpowiednią oś za pomocą funkcji $Projection$, a następnie wybieramy najlepszą hiperpłaszczyznę równoległą do podprzestrzeni liniowej 
reprezentowanej przez osobnika za pomocą funkcji oceny. Na rysunku 4.1 widzimy to dla przestrzeni dwuwymiarowej.
\begin{figure}
 \caption{Interpretacja szukania optymalnej hiperpłaszczyzny w przestrzeni dwuwymiarowej}
 \includegraphics[scale=0.8]{hiperp.png}
\end{figure}
Powtarzając algorytm dla każdej z osi,
wybierzemy najlepszą hiperpłaszczyznę. Hiperpłaszczyzny szukamy do momentu, w którym tabela otrzymana z nowych cech indykatorów odpowiednich 
półprzestrzeni będzie spójna, bądź spełniony zostanie warunek stopu, o którym opowiem w następnych rozdziałach. Na rysunku 4.2 widac 
przykładowy podział obiektów
w przestrzeni dwuwymiarowej po znalezieniu dwóch hiperpłaszczyzn.
\begin{figure}
 \caption{Podział obiektów przez hiperpłaszczyzny}
 \includegraphics[scale=0.8]{obszary.png}
\end{figure}
Po znalezieniu kolejnych hiperpłaszczyzn grupy obszarów niespójnych będą się stale zmniejszać. Funkcję liczę jedynie dla obszarów
niespójnych, które nie są jeszcze podzielone. Funkcję celu dla listy obszarów niespójnych wyliczam z wzoru (kod w języku python):
\begin{python}
# unconsistent_groups jest lista niespojnych grup obiektow
award = sum([discMeasure(H, U) for U in unconsistent_groups])
\end{python}
Kod algorytmu w języku python:
\begin{python}
# dla wszyskich funkcji:

# dec <- lista decyzji
# table <- tabela 
# attrs_list <- lista atrybutow

# ekstracja nowych cech, generowanie nowej tabeli
def extract_features(sc=None):
        extracted_table = []
        while True:
            unconsistent_groups = ConsistentChecker.
				  count_unconsistent_groups(
				  extracted_table, dec, sc)
# kompresja niespojnych obszarow, np wyrzucanie 
# bliskch punktow o roznych decyzjach
            unconsistent_groups = filter(
				  lambda x: 
				  dpoints_strategy.decision(x) is None, 
				  unconsistent_groups)
# jesli sa jeszcze jakies niespojne grupy szukamy hiperplaszczyzn
            if unconsistent_groups:
            
                best_hyperplane = search_best_hyperplane(
				  unconsistent_groups, sc)
                extracted_table.append(
				count_objects_positions(best_hyperplane))
            else:
                break

        return np.transpose(np.array(extracted_table))

# szukanie optymalnej hiperplaszczyzny
def search_best_hyperplane(unconsistent_groups, sc=None):

            rdd_attributes = sc.parallelize(attrs_list)
# zrownoleglamy ze wzgledu na osie (atrybuty)
# i liczymy optymalne hiperplaszczyzny dla rzutow
            hyperplanes = rdd_attributes.map(
			  lambda x: 
			  search_best_hyperplane_for_projection(
			  x, unconsistent_groups)).collect()
	
# zwracamy najlepsza
        return max(hyperplanes, key=lambda x: x[1][0])

# szukanie optymalnej hiperplaszczyzny dla rzutu
def search_best_hyperplane_for_projection(
    attr, unconsistent_groups, sc=None):

        projection_axis = table[attr]
        other_axes = [x for x in attrs_list if not x == attr]
        new_table = table[other_axes]
	
	# szukamy optymalnej hiperplaszczyzny algorytmem genetycznym
        gen_search = GeneticSearch(len(other_axes), dec, new_table,
                                   projection_axis, unconsistent_groups)
        cand_hyperplane = gen_search.genetic_search(sc)

        return attr, cand_hyperplane

# unconsistent_groups <- lista niespojnych grup
# population_size <- rozmiar populacji
# przeszukiwanie genetyczne
def genetic_search(sc=None):

# inicjowanie populacji
        population = init_generation()

        current_best_award = 0
        the_same_awards = 0
        for i in range(max_iter):
# liczymy funkcje celu dla populacji
            rdd_population = sc.parallelize(population, 
					    population_size * 10)
            awards = rdd_population.mapPartitions(
				    count_award_for_chunk).collect()
	    
# wybieramy najlepszych osobników
            population_awards = select_best_individuals(awards)

            best_individual = population_awards[0]

            population = map(lambda x: x[1], population_awards)

            new_generation = count_new_generation(
			     copy.deepcopy(population))

            population = new_generation + population

        return best_individual
        
def count_award_for_chunk(self, population):
        for individual in population:
            yield self.count_award(individual)
            
def count_award(individual):
        all_objects = reduce(add, unconsistent_groups)
# liczymy rzuty wszystkich obiektow
        projections = count_projections(individual, all_objects)
# sortujemy po wartosci rzutu
        objects = sorted(projections, key=lambda x: x[1])

# liczenie funkcji oceny (discMeasure) dla posortowanych obiektow
        object_group_dict = {}
        group_fqs_dict = {}
        act_left_sum = {}
        act_right_sum = {}

        for ind, group in enumerate(unconsistent_groups):
            group_decisions = []
            for obj in group:
                group_decisions.append(self.dec[obj])
                object_group_dict[obj] = ind
            act_left_sum[ind] = 0
            act_right_sum[ind] = len(group_decisions)
            group_fqs_dict[ind] = DiscMeasureCalculator.
				 prepare_hist(group_decisions)

        act_award = 0
        max_award = 0
        for obj, proj in objects:
            group_id = object_group_dict[obj]
            dec = self.dec[obj]
            act_left_sum[group_id], act_right_sum[group_id], act_award = \
                DiscMeasureCalculator.
	        update_award(dec, act_left_sum[group_id], 
			     act_right_sum[group_id],
                             group_fqs_dict[group_id], act_award)
            if act_award > max_award:
                max_award = act_award
                good_proj = proj
                
# zwracamy wartosc najlepszego wyniku, osobnika (wektory) i przesuniecie
        return max_award, individual, good_proj  

# wyliczanie rzutu dla osobnika i grupy obiektow 
def count_projections(individual, objects):
        table = table_as_matrix[objects, :]
        proj = [projection_axis[obj] for obj in objects]
        return zip(objects, proj - table.dot(individual))
\end{python}

\section{Indukcja drzewa decyzyjnego na podstawie cięć przez hiperpłaszczyzny}
Optymalne hiperpłaszczyzny mogą być użyte nie tylko ekstracji nowych cech, ale również do indukcji drzewa
decyzyjnego. W standardowym drzewie, w węźle mamy funkcję $test$ odpowiadającą atrybutom, dla wartości ciągłych zwykle
jest to:
\begin{align*}
test(x) = \begin{cases} 1 \text{ jeśli } a_i(x) > c \\ 0 \text{ wpp } \end{cases}
\end{align*}
Możemy jednak użyć funkcji testu opartej o hiperplaszczyzny, wtedy drzewo będzie miało zwykle mniejszą głębokośc
(hiperpłaszczyzny mogą mieć wyższą wartośc funkcji oceny na danym węźle), ale większy jest koszt czasowy predykcji
(funkcja testu działa w czasie $O(n)$ gdzie $n$ jest liczbą atrybutów) oraz szukania drzewa. Funkcja testu dla 
uogólnionego drzewa (dla wag $w_i$):
  \begin{align*}
test(x) = \begin{cases} 1 \text{ jeśli } w_1 a_1(x) + ... + w_n a_n(x) > c \\ 0 \text{ wpp } \end{cases}
\end{align*}
Oraz funkcja dla testu dla reprezentacji hiperplaszczyzny z algorytmu genetycznego (gdy $x_1$ jest najlepszą osią):
\begin{align*}
 test(x) = \begin{cases} 1 \text{ jeśli } a_1(x) - \alpha_{j_2}^2 a_2(x) - \alpha_{j_3}^3 a_3(x) - 
 ... - \alpha_{j_n}^n a_n(x) \geq p \\ 0 \text{ wpp } \end{cases}
\end{align*}
Algorytm znajdowania drzewa przez algorytm genetyczny w języku python:
\begin{python}
def count_decision_tree(objects, sc=None):
# jesli mozemy podjac decyzje dla danego zbioru obiektow
# zwracamy koncowy wezel
        decision = dpoints_strategy.decision(objects)
        if decision is not None:
            return DecisionTree(decision, 0, 0)

# szukamy najlepszej hiperplaszczyzny algorytmem genetycznym
# (tak jak dla ekstrakcji cech)
        best_hyperplane = search_best_hyperplane([objects], sc)
        hyperplane_indicator = count_objects_positions(
			       best_hyperplane, objects)
        left_son_objects = map(
			   lambda x: x[1], 
			   filter(
			   lambda (i, x): 
			   hyperplane_indicator[i] == 0, 
			   enumerate(objects)))
        right_son_objects = map(
			    lambda x: x[1], 
			    filter(
			    lambda (i, x): 
			    hyperplane_indicator[i] == 1, 
			    enumerate(objects)))

        return DecisionTree(best_hyperplane, 
			    count_decision_tree(left_son_objects),
                            count_decision_tree(right_son_objects))
\end{python}

\subsection{Generowanie lasu losowego}
Jako, że w algrytmie genetycznym początkowe wartości wektorów dobierane są w sposób losowy, 
drzewa decyzyjne generowane kilkukrotnie przez algorytm genetyczny mogą się od siebie różnić.
Przy jednokrotnym generowaniu drzewa istnieje prawdopodobieństwo utknięcia w lokalnym maksimum funkcji celu.
Możemy więc stworzyć wiele drzew i podjąć decyzję na podstawie głosowania, tworząc coś na kształt lasu losowego.
Algorytm w języku python:
\begin{python}
# objects <- lista wszyskich obiektow w tabeli
def count_random_forest(size=50):	
	return [count_decision_tree(objects) for _ in range(size)]
      
def predict(object):
	forest = count_random_forest()
# liczymy predykcje dla kazdego drzewa
	decisions = map(lambda tree: tree.predict(object), forest)
# zwracamy najczesciej wystpujaca decyzje
	return most_common(Counter(decisions))		
\end{python}

\section{Drzewo SVM}
Dla celów analizy porównawczej możemy stworzyć drzewo, w którym funkcją testu
również będzie funkcja charakterystyczna półprzestrzeni, ale odpowiednią hiperpłaszczyznę
będziemy szukać nie za pomocą algorytmu genetycznego, ale algorytmu SVM, szukającego 
hiperpłaszczyzny maksymalizującej sumę odległości od obiektów. Rysunek 4.3 ilustruje działanie
algorytmu SVM.
\begin{figure}
 \caption{Podział obiektów przez hiperpłaszczyznę SVM}
 \includegraphics[scale=0.8]{svm_hiperp.png}
\end{figure}
Kod algorytmu w języku python:
\begin{python}
def count_decision_tree(objects, sc=None):
        decision = dpoints_strategy.decision(objects)
        if decision is not None:
            return DecisionTree(decision, 0, 0)

        X = [self.table[i] for i in objects]
        y = [self.dec[i] for i in objects]
        svm = LinearSVC()
# wyliczanie SVM z zewnetrznej biblioteki
        svm.fit(X, y)
# wybieranie wspolczynnikow optymalnej hiperplaszczyzny
        coefs = svm.coef_[0]
        intercept = svm.intercept_[0]
        best_hyperplane = (intercept, coefs)
        hyperplane_indicator = map(
			       lambda r: 
			       (np.dot(
			       list(r), coefs) + intercept > 0),
			       X)
        left_son_objects = map(
			   lambda x: x[1], 
			   filter(
			   lambda (i, x): 
			   not hyperplane_indicator[i], 
			   enumerate(objects)))
        right_son_objects = map(
		            lambda x: x[1], 
		            filter(lambda (i, x): 
		            hyperplane_indicator[i], 
		            enumerate(objects)))

        return DecisionTree(best_hyperplane, 
			    count_decision_tree(left_son_objects),
                            count_decision_tree(right_son_objects))
\end{python}

\section{Strategie klasyfikacji bliskich punktów o różnej decyzji}
Zarówno w algorytmie indukcji drzewa decyzyjnego jak i w algorytmie szukania 
nowych cech pojawiła się funkcja $dpoints\_strategy.decison(objects)$. W przypadku
gdy pojawią się punkty będące ``blisko'' siebie mające różne decyzje algorytmowi genetycznemu ciężko
będzie ``wcisnąć'' się pomiędzy te punkty przy losowej optymalizacji. Dlatego w przypadku indukcji drzewa 
wyliczamy decyzję dla takich punktów w inny, deterministyczny sposób, natomiast przy generowaniu cech
takie punkty ignorujemy. Pytanie jakie się nasuwają są dwa:
\begin{enumerate}
 \item Kiedy uznawać, że punkty są bliskie?
 \item W jaki sposób podejmować decyzję dla takich punktów?
\end{enumerate}
W swojej pracy uznaję punkty za bliskie, gdy średnica zbioru jest mniejsza niż pewien procent 
średnicy całego zbioru (problem pojawia się gdy występują outliery, wtedy można liczyć średnicę
bez outlierów). Decyzją zaś jest najczęściej występująca decyzja w danym zbiorze. Kod algorytmu w języku python:
\begin{python}
# max_dist <- srednia calego zbioru obiektow
def decision(objects, ratio):
	max_dist_objects = count_max_dist(objects)
	if max_dist_objects / max_dist < ratio:
	  return most_common(Counter(decisions(objects)))
	return None
\end{python}


\section{Wyniki eksperymentów}

\begin{thebibliography}{9}

\bibitem{Discretization}
  Hun Son Nguyen,
  \emph{Approximate Boolean Reasoning: Foundations and Applications in Data Mining}.
  Institute of Mathematics, Warsaw University,
  1998.
\bibitem{Hyperplanes}
  Hun Son Nguyen,
  \emph{From Optimal Hyperplanes to Optimal Decision Trees}.
  Institute of Mathematics, Warsaw University,
  1998.
\bibitem{BooleanGenetic}
  Srilatha Chebrolu, Sriram G Sanjeevi,
  \emph{Rough set theory for discretization based on boolean reasoning and 
  genetic algorithm}.
  Department of Computer Science and Engineering, NIT Warangal, India,
  2012.
\bibitem{SparkDocumentation}
 Spark Documentation,
 \emph{http://spark.apache.org/docs/latest/programming-guide.html}

\end{thebibliography}
 
\end{document}