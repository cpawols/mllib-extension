\documentclass[magisterska]{pracamgr}
\usepackage[usenames,dvipsnames]{xcolor}

\usepackage[MeX,plmath]{polski}
\usepackage[utf8]{inputenc}
\usepackage{amsmath}
\usepackage{graphicx}
\usepackage{amsfonts}
\usepackage{amsthm}
\usepackage{enumerate}
\usepackage{color}
\usepackage{amsfonts}
\usepackage{fancyhdr}
\usepackage{amssymb}
\usepackage{amsthm}
\usepackage{textcomp}
\usepackage{tikz}
\usepackage{hyperref}
\usepackage[toc,page]{appendix}
\usepackage[font=large,labelfont=bf]{caption}

\usepackage{t1enc}
\usepackage{a4wide}
\usepackage{amsfonts}
\usepackage{amsmath}
\usepackage{enumerate}
\usepackage{verbatim}
\usepackage{geometry}
\geometry{left=1in,right=1in,%
bindingoffset=0mm, top=1in, bottom=1in}
\usepackage{amssymb, latexsym}
\usepackage{amsthm}
\usepackage{palatino}
\usepackage{array}
\usepackage[]{algorithm2e}
%\usepackage{pstricks}
\usepackage{enumitem}
\setlist[enumerate]{label*=\arabic*.}
\usepackage{textcomp}
\usepackage{listings}
\usepackage[export]{adjustbox}
\lstset{language=Python}

\usepackage[]{algorithm2e} %Algorytm czy Algorithm?
\renewcommand{\listalgorithmcfname}{Lista algorytmów}%
\renewcommand{\algorithmcfname}{Algorytm}%
\renewcommand{\algorithmautorefname}{algorytm}%
\renewcommand{\algorithmcflinename}{linia}%
\makeatletter
\renewcommand{\algocf@typo}{}%
\renewcommand{\@algocf@procname}{Procedura}%
\renewcommand{\@algocf@funcname}{Funkcja}%
\makeatother
\renewcommand{\procedureautorefname}{procedura}%
\renewcommand{\functionautorefname}{funkcja}%

\lstdefinestyle{customc}{
  belowcaptionskip=1\baselineskip,
  breaklines=true,
  frame=L,
  xleftmargin=\parindent,
  language=python,
  showstringspaces=true,
  basicstyle=\large\ttfamily,
  keywordstyle=\bfseries\color{green!40!black},
  commentstyle=\itshape\color{black},
  identifierstyle=\color{blue},
  stringstyle=\color{orange},
}

\lstdefinestyle{customasm}{
  belowcaptionskip=1\baselineskip,
  frame=L,
  xleftmargin=\parindent,
  language=[x86masm]Assembler,
  basicstyle=\footnotesize\ttfamily,
  commentstyle=\itshape\color{purple!40!black},
}

\lstset{escapechar=@,style=customc}

% Tu jest dobre miejsce na Twoje wżasne makra i~żrodowiska:
\newtheorem{thm}{Thm}[chapter]
\theoremstyle{plain}
\newtheorem{twierdzenie}[thm]{Twierdzenie}
\newtheorem{stwierdzenie}[thm]{Stwierdzenie}
\newtheorem{wniosek}[thm]{Wniosek}
\newtheorem{lemat}[thm]{Lemat}
\newtheorem{przyklad}[thm]{Przykład}

\theoremstyle{definition}
\newtheorem{definicja}[thm]{Definicja}

\theoremstyle{remark}
\newtheorem{uwaga}[thm]{Uwaga}
\newtheorem{zadanie}{Zadanie}

\renewcommand{\Re}{\mathrm{Re}}
\renewcommand{\Im}{\mathrm{Im}}
\newcommand{\wek}{\begin{pmatrix} k \\
		l \end{pmatrix}}

\newcommand{\wekx}{\begin{pmatrix} x \\
		y \end{pmatrix}}		

\newcolumntype{P}[1]{>{\centering\arraybackslash}p{#1}}

% koniec definicji

\author{Krzysztof Rutkowski}

\nralbumu{319379}

\title{Wybrane metody dyskretyzacji i generowania nowych cech z zastosowaniem paradygmatu MapReduce}

\tytulang{Selected discretization and feature extraction methods with application of MapReduce paradigm}

\kierunek{Matematyka}


\opiekun{prof.dr.hab Andrzej Skowron\\
  Zakład Logiki Matematycznej\\
  }

\date{Czerwiec 2016}

\dziedzina{ 
11.0 Matematyka i Informatyka\\
11.4 Sztuczna inteligencja
}

%Klasyfikacja tematyczna wedlug AMS (matematyka) lub ACM (informatyka)
\klasyfikacja{68T Artificial Intelligence, 68T05 Learning and adaptive systems}

\keywords{discretization, feature extraction, decision system, mapreduce}

\begin{document}
\maketitle

%tu idzie streszczenie na strone poczatkowa
\begin{abstract}
Praca przedstawia przegląd metod dyskretyzacji oraz generowania nowych cech i indukcji drzew decyzyjnych przez hiperpłaszczyzny. 
Zaimplementowane algorytmy zbadano pod kątem 
skuteczności oraz skalowalności. Do obliczeń równoległych wykorzystano paradygmat obliczeniowy MapReduce.
Rozdziały \ref{r:pojecia} oraz \ref{r:skalowalnosc} zostały spisane wspólnie z Pawłem Olszewskim.
\end{abstract}

\tableofcontents
%\listoffigures
%\listoftables


%\chapter{Wprowadzenie}
\chapter{Podstawowe informacje}\label{r:pojecia}
Definicje w tym rozdziale zostały zaczerpnięte z \cite{Rudiments} oraz \cite{Discretization}.
\begin{definicja}
Systemem decyzyjnym ($ang. \ Decision \ System$) nazwiemy następującą
trójkę $\mathbb{S} = (U, A, d)$, gdzie
\begin{itemize}
 \item U jest skończonym zbiorem, którego elementami są obiekty:
    \begin{align*}
	    U = \{u_1, \ldots, u_n \},  
    \end{align*}

 \item A jest skończonym zbiorem atrybutów warunkowych (cech) takim, że jeśli $ a \in A$ to
    \begin{align*}
	  a:U \rightarrow V^{a},
    \end{align*}
    gdzie przez $V^{a}$ oznaczamy zbiór wartości atrybutów,
    
 \item zaś d nazywamy atrybutem decyzyjnym
    \begin{align*}
     d: U \rightarrow \{d_1, \ldots d_k\},
    \end{align*}
  gdzie zbiór $\{d_1, \ldots d_k\}$, jest to zbiór wszystkich możliwych decyzji.
\end{itemize}

\end{definicja}

Jedną z możliwości reprezentacji systemu decyzyjnego  \ref{SystemDecyzyjny:1.1} jest dwuwymiarowa tabela.
W poszczególnych wierszach tabeli znajdują się obiekty $\{u_1, \ldots, u_n \}$,
natomiast w kolumnach znajdują się wartości atrybutów $\{a_{1}, \ldots, a_{k} \}$.
Dokładniej mówiąc w polu $i,$ odpowiadającemu $i$-temu wierszowi i $j$-tej kolumnie
znajduje się wartość $j$-tego atrybutu ($a_{j}$) dla obiektu $u_{i}$.
W ostatniej kolumnie umieszczony jest atrybut decyzyjny $d_{i}$ dla obiektu $u_i$.

\begin{center}
  \begin{tabular}{  l || l l l | r }
  \label{SystemDecyzyjny:1.1}
    $U$ & $a_1$ & $\ldots$ & $a_n$ & $d$ \\ 
    \hline
  
    $u_1$ & $a_{11}$ & $\ldots $ & $a_{1n} $ & $d_{1}$ \\
    $u_2$ & $a_{21}$ & $\ldots $ & $a_{2n} $ & $d_{2}$ \\ 
    \ \ $\vdots$ & \ \ $\vdots$ & \ \ $\vdots$ & \   $\vdots$ & $\vdots$ \\ 
    $u_n$ & $a_{n1}$ & $\ldots $ & $a_{nn} $ & $d_{n}$ \\     
  \end{tabular}\par
  \bigskip 
  \ref{SystemDecyzyjny:1.1} : System decyzyjny

\end{center}

Obiektami tablicy decyzyjnej mogą być np. szeregi czasowe, ludzie,
papiery wartościowe. \newline Dla przykładu rozważmy tablicę decyzyjną,
której obiektami są ludzie. Rozważmy pacjentów, którzy zostali przebadani pod kątem białaczki.
Każdemu pacjentowi zbadano kilkanaście genów. Wynikiem tego padania jest kilkanaście 
liczb rzeczywistych opisujących poszczególne geny. Wartość decyzji oznacza diagnozę pacjenta.
\newline
\begin{center}
 \begin{tabular}{l || P{2.5cm} P{2.5cm} P{2.5cm} | P{1.5cm}}
  $Pacjent$     & gen 1 & gen 2 & gen 3 & Chory \\ 
  \hline
  $pacjent_{1}$ & 0.02       & 0.12       & 1.2    & Tak \\
  $pacjent_{2}$ & -0.45      & 4.56       & 2.3    & Nie \\
  $pacjent_{3}$ & 1.32       & 2.3        & 0.01   & Tak \\
  $pacjent_{4}$ & 1.23       & 1.2        & 2.12   & Nie \\
 \end{tabular}
\end{center}
W powyższej sytuacji obiekt (pacjent) tablicy decyzyjnej zawiera ciąg liczb (każda traktowana jako
jeden atrybut), które zostały uzyskane podczas przeprowadzonego badnia. Ostatnim elementem wiersza
jest atrybut decyzyjny(decyzja), stwierdzająca czy pacjent jest chory.
\newline
Często zdarza się tak, że w tabeli występują wartości rzeczywiste, np. w powyższych danych medycznych, gdzie 
wyniki badań mają wartości z pewnego przedziału bądź gdy mamy podane pewne ciągłe wielkości fizyczne, np.
temperaturę. Z różnych względów (np wymagają tego niektóre klasyfikatory) chcieliśmy mieć tabelę z wartościami
nominalnymi. I tak, np. poniższą tabelę:

\begin{center}
 \begin{tabular}{l || P{2.5cm} P{2.5cm} P{2.5cm} | P{1.5cm}}
  $U$     & Temparatura & Wilgotność & Zachmurzenie & Gra \\ 
  \hline
  $u_{1}$ & 0      & 30      & słabe          & Nie \\
  $u_{2}$ & 10     & 90      & umiarkowane    & Nie \\
  $u_{3}$ & 20     & 30      & słabe   	      & Tak \\
  $u_{4}$ & 15     & 30      & deszcz         & Nie \\
 \end{tabular}
\end{center}

moglibyśmy zastąpić tabelą:

\begin{center}
 \begin{tabular}{l || P{2.5cm} P{2.5cm} P{2.5cm} | P{1.5cm}}
  $U$     & Temparatura & Wilgotność & Zachmurzenie & Gra \\ 
  \hline
  $u_{1}$ & niska      & mała      & słabe          & Nie \\
  $u_{2}$ & średnia    & duża      & umiarkowane    & Nie \\
  $u_{3}$ & wysoka     & mała      & słabe   	    & Tak \\
  $u_{4}$ & średnia    & mała      & deszcz         & Nie \\
 \end{tabular}
\end{center}

\newpage
Wprowadzę teraz wstępne pojęcia dotyczące dyskretyzacji atrybutów rzeczywistych.


\begin{definicja}[Niesprzeczna tabela decyzyjna]
Tabela  $\mathbb{S} =(U, A, d)$  jest niesprzeczna, jeśli dla dowolnych dwóch obiektów $u_{1}$ i $u_{2}$:
  \begin{align*}
     (a_{1}(u_{1}), a_{2}(u_{1}), ..., a_{k}(u_{1})) = (a_{1}(u_{2}), a_{2}(u_{2}), ..., a_{k}(u_{2})) \Rightarrow 
     d(u_{1}) = d(u_{2}),
  \end{align*}
tzn. jeśli mamy te same wartości na atrybutach wartości atrybutów decyzyjnych również powinny być takie same.
\end{definicja}

\begin{definicja}[Podział atrubutu]
Dla tabeli  $\mathbb{S} =(U, A, d)$  i zbioru cięć dla atrubutu $a$:
\begin{align*}
     c_{0}^{a} < c_{1}^{a} < ... < c_{k_{a}}^{a} < c_{k_{a} + 1}^{a},
  \end{align*}
definiujemy podział (dyskretyzacje) atrubutu $a$ jako:
  \begin{align*}
     P_{a} = {[c_{0}^{a}, c_{1}^{a}], ... , [c_{k_{a}}^{a}, c_{k_{a} + 1}^{a}]}.
  \end{align*}
Zaś podział całego uniwersum jako:
  \begin{align*}
     P = \cup_{a \in A} P_{a}.
  \end{align*}
\end{definicja}

\begin{definicja}[Zdyskretyzowana tabela decyzyjna]
 Zdyskretyzowaną tabelą decyzyjną $\mathbb{S} =(U, A', d)$ otrzymaną z tabeli $\mathbb{S} =(U, A, d)$ przez podział
 $P$ jest tabela, dla której jeśli $a' \in A'$ to
 \begin{align*}
     a'(u_k) = l,
  \end{align*}
  gdzie 
  \begin{align*}
   c_{l}^{a} \leq a(u_k) < c_{l+1}^{a},
  \end{align*}
  oraz
  \begin{align*}
     P_{a} = {[c_{0}^{a}, c_{1}^{a}], ... , [c_{k_{a}}^{a}, c_{k_{a} + 1}^{a}]}.
  \end{align*}
\end{definicja}
Atrybut $a'$ jest nowym atrybutem otrzymanym poprzez dyskretyzację atrubutu $a$.

\begin{definicja}[Niesprzeczny zbiór cięć]
 Zbiór cięć $P$ jest niesprzeczny gdy otrzymana z niego tabela decyzyjna $\mathbb{S} =(U, A', d)$ z tabeli $\mathbb{S} =(U, A, d)$
 jest niesprzeczna.
\end{definicja}

\begin{uwaga}
Szukanie niesprzecznego zbioru cięć o najmniejszej mocy jest problemem $NP$-trudnym.
Dowód poprzez sprowadzenie do problemu \textbf{SET COVER} znajdziemy w \cite{Discretization}.
\end{uwaga}


\chapter{Skalowalność}\label{r:skalowalnosc}

Większość algorytmów została zaimplementowana w sposób skalowalny za pomocą frameworku 
dla obliczeń równoległych \textbf{Apache Spark} (\cite{SparkDocumentation}) bazującego na paradygmacie MapReduce (\cite{MapReduce}).
Na wstępie przytoczę kilka pojęć przydatnych w dalszej części omawiania Sparka.

\section{Pojęcia wstępne}
\begin{itemize}
 \item \textbf{Programowanie funkcyjne}
Filozofia i metodyka programowania będąca odmianą programowania deklaratywnego, 
w której funkcje należą do wartości podstawowych, a nacisk kładzie się na wartościowanie 
(często rekurencyjnych) funkcji, a nie na wykonywanie poleceń.
W czystym programowaniu funkcyjnym, raz zdefiniowana funkcja zwraca zawsze 
tę samą wartość dla danych wartości argumentów, tak jak funkcje matematyczne.
W czystych językach funkcyjnych nie występują zmienne ani efekty uboczne, 
a wartościowanie jest leniwe, tj wykonywanie obliczenia następuje tylko w momencie, gdy jest to potrzebne.

\item \textbf{Klaster, skupisko}
Grupa połączonych jednostek komputerowych, które współpracują ze sobą w celu udostępnienia zintegrowanego środowiska pracy.
Komputery wchodzące w skład klastra (będące członkami klastra) nazywamy węzłami (ang. node).

\item \textbf{HDFS}
Hadoop Distributed File System, rozproszony, skalowalny, spójny system plików używany przez framework Apache Spark.

\item \textbf{RDD}
Resilent Distributed Dataset, podstawowa struktura danych w Sparku. RDD jest kolekcją elementów dystrybuowaną pomiędzy 
węzły klastra, na których operacje mogą być wykonywane równolegle. Po zakończenie operacji na RDD można je dalej 
przechowywać w pamięci RAM, co umożliwia szybsze dostęp do następnych obliczeń. 
\end{itemize}




\section{Paradygmat MapReduce}

Paradygmat MapReduce został wprowadzony przez firmę \textbf{Google} i służy do równoległego przetwarzania
dużych danych przy użyciu klastra.
Rozproszenie obliczeń pomaga w istotny sposób zmniejszyć czas wykonywanych zadań.
Apache Spark jest niejako rozszerzeniem wprowadzonego paradygmatu. 
Jak każdy framework do obliczeń na klastrze posiada pod sobą rozproszony system plików (HDFS), 
a także oprogramowanie (dla workerów i mastera), które odpowiada za zarządzanie awariami, przydzielanie 
zadań i organizację. 
\newpage

\begin{figure}
 \caption{architektura Sparka}
 \includegraphics[scale=0.8]{spark-architecture.png}
\end{figure}

Spark jest dedykowany dla języka funkcyjnego - Scali, algorytmy zostały jednak napisane w pythonie, który 
mimo, iż językiem funkcyjnym nie jest, posiada jednak cechy języków funkcyjnych. 

Typowy potok obliczeń w Sparku polega na:
\begin{itemize}
 \item stworzeniu RDD bazującego na kolekcji utrzymywanej w programie,
 \item wykonaniu transformacji na RDD,
 \item wykonaniu akcji.
\end{itemize}

Funkcje w Sparku dzielą się na transformacje i akcje:
\begin{itemize}
 \item \textbf{transformacja} jest funkcją działającą na RDD i zwracającą RDD,
 \item \textbf{akcja} jest funkcją działającą na RDD i zwracającą kolekcję do programu po zakończeniu obliczeń.
\end{itemize}
Zarówno akcje, jak i transformacje wykonywane są leniwie. Przykłady transformacji:
\begin{itemize}
 \item \textbf{map(f)} - zwraca nowy RDD aplikując funkcję $f$ do każdego elementu RDD,
 \item \textbf{filter(f)} - zwraca nowy RDD składający się tylko z tych elementów, dla których funkcja $f$ zwróci True,
 \item \textbf{reduceByKey(f)}\label{reduceByKey} - dla RDD postaci (klucz, wartość), grupuje według kluczy i zwraca RDD z agregacją wartości 
 za pomocą funkcji $f$ dla zgrupowanych kluczy.
\end{itemize}

Przykłady akcji:
\begin{itemize}
 \item \textbf{reduce(f)} - zwraca kolekcję będą agregacją RDD za pomocą funkcji $f$, która przyjmuje dwa 
 argumenty i zwraca jeden,
 \item \textbf{collect()} - zwraca kolekcję elementów do programu.
\end{itemize}

W powyższych przykładach zamiast funkcji $f$ można użyć wyrażeń lambda (funkcji anonimowych).

Przykład, zliczanie słów w języku python:
\begin{lstlisting}
#ustawiamy kontekst i konfiguracje
conf = SparkConf().
       setMaster("spark://localhost:7077").
       setAppName("entropy")
sc = SparkContext(conf=conf)
words = ["ala", "ma", "ala", "kota", "ma"]
#tworzenie rdd na podstawie listy
words_rdd = sc.paralellize(words)  
words_counts =  words_rdd.
		map(lambda x: (x, 1)).
		reduceByKey(lambda x, y: x+y)
\end{lstlisting}

W powyższym, jak i w następnych listingach algorytmów w języku python, sc zawsze będzie obiektem klasy \textbf{SparkContext}, który inicjuje
RDD na podstawie konfiguracji klastrowego środowiska dla Sparka. Konfigurację tworzymy na podstawie obiektu klasy \textbf{SparkConf}, bądź
ustawiamy parametry klastra, takie jak:
\begin{itemize}
 \item liczba komputerów,
 \item liczba procesorów,
 \item pamięć RAM na jednostkę obliczeniową,
 \item nazwa aplikacji,
\end{itemize}
w odpowiednim pliku konfiguracyjnym. 


Obliczenia wykonywane w Sparku są nie tylko skalowalne, ale również szybkie ze względu na:
\begin{itemize}
 \item leniwość wykonywanych operacji, obliczenia wykonywane są tylko wtedy gdy są potrzebne (ważna cecha
 języków funkcyjnych),
 \item możliwość przechowywania rozproszonych struktur danych (RDD) w pamięci RAM, co powoduje szybszy dostęp do danych
 podczas ponownego przetwarzania (np w Hadoop MapReduce musieliśmy dane wczytywać z HDFS).
\end{itemize}


\begin{figure}
 \caption{porównanie szybkości działania Hadoop MapReduce i Spark na przykładzie regresji logistycznej}\label{fig:hadoop_spark}
 \includegraphics[scale=1.0]{./wykresy/hadoop_spark.png}
\end{figure}

Na rysunku \ref{fig:hadoop_spark} widzimy porównanie szybkości działania Hadoop MapReduce i Sparka. Widzimy jak bardzo idea RDD, która 
nie występuje w Hadoop MapReduce zwiększa szybkość działania.

\section{Metryki}
Do oceniania jakości skalowalnych algorytmów wykorzystywanych jest kilka metryk pozwalających
 ocenić czy dany algorytm skaluje się w odpowiedni sposób. W tej sekcji zaprezentowane zostały przykłady
 metryk pozwalających ocenić skalowalność algorytmów.
 
\textbf{Speed\_up}
 
Jedną z metryk mierzącą ile razy obliczenia na jednej maszynie trwają dłużej od
obliczeń wykonywanych na \emph{n} maszynach jest metryka zdefiniowana następująco:
 
\begin{align*}
 speed\_up(n) = \frac{t_1}{t_n},
\end{align*}
gdzie $t_j$ jest czasem obliczeń wykonywanych na $j$ maszynach.
 
\textbf{Scale\_up}
 
Ta metryka z kolei opisuje jak zmienia się czas obliczeń podczas gdy
liczba komputerów oraz wielkość zbioru danych rosną taką samą liczbę razy.
 
\begin{align*}
 scale\_up(n) = \frac{t_{D_n}}{t_{D_1}},
\end{align*}
gdzie:
\begin{itemize}
 \item $t_{D_1}$ -- czas obliczeń dla ustalonego zbioru danych na jednym komputerze,
 \item $t_{D_n}$ -- czas obliczeń dla $n$ --razy większego zbioru danych na $n$ komputerach.
\end{itemize}

\chapter{Biblioteka mllib-extensions}
Algorytmy użyte w dalsze części pracy zostały zaimplementowane w bibliotece \textbf{mllib-extensions},
umieszczonej w repozytorium, które znajduje się pod adresem: \url{https://github.com/cpawols/mllib-extension}.
Biblioteka została napisana w języku python,
jest skalowalna, wszystkie obliczenia mogą być wykonywane na wielu maszynach, bądź w zwyczajnym trybie, 
na jednym procesorze, zostały zaimplementowane następujące moduły:
  \begin{itemize}
   \item \textbf{discretization}: moduł do wykonywania dyskretyzacji,
   \item \textbf{feature-extraction}: moduł do generowania nowych cech,
   \item \textbf{decision-tree}: moduł do indukcji drzew decyzyjnych,
   \item \textbf{reduct-feature-selection}: moduł do selekcji atrybutów (autor: Paweł Olszewski).
  \end{itemize}


\chapter{Skalowalne algorytmy dyskretyzacji}

W tym rozdziale omówię zaimplementowane algorytmy dyskretyzacji, jak i przedstawię wyniki eksperymentów i obliczeń.
Algorytmy dyskretyzacji dzielimy na:
\begin{itemize}
 \item \textbf{globalne} - poszukiwane są cięcia na podstawie całej tabeli, tj na podstawie zagregowanej informacji ze wszystkich pól tabeli
 \item \textbf{lokalne} - poszukiwane są cięcia na podstawie każdego atrubutu niezależnie, wykorzystując jedynie informacje w kolumnach
\end{itemize}
Będziemy tu analizować algorytmy lokalne, gdyż
w naturalny sposób łatwo je zrównoleglić, zgodnie ze schematem:
\begin{lstlisting}
# tworzymy RDD na podstawie listy kolumn i 
# dzielimy na numer_of_columns kawalkow
columns_rdd = sc.paralellize(table, numer_of_columns)
# wykonujemy algorytm dyskretyzacyjny rownolegle dla kazdej kolumny
discretized_columns = columns_rdd.
		      mapPartitions(discretizeColumn).
		      collect()
\end{lstlisting}
W algorytmie tym \textbf{discretizeColumn} przyjmuje kolumnę liczb rzeczywistych tabeli decyzyjnej i zwraca
jej dyskretyzację. Algorytm równolegle oblicza dyskretyzację dla każdej z kolumn.
Metody globalne szukają zbioru cięć dla całej tabeli, przez co trudniej je zrównoleglić.
Niżej przedstawię 3 lokalne algorytmy dyskretyzacji.

\section{Sprawdzanie spójności tabeli}
W przypadku sensownej dyskretyzacji należy sprawdzić czy tabela jest spójna. W Sparku możemy 
to w prosty sposób zrobić za pomocą poniższego algorytmu:
\begin{lstlisting}
# funkcja agregujaca, zwraca tuple skladajaca sie z roznych decyzji
# dla danej kombinacji atrybutow
def get_unique_dec(x, y):
  return tuple(set(x + y))
# rows_rdd = (row, (dec,))
rows_rdd = sc.paralellize(table, numer_of_rows)
one_dec = rows_rdd.
	  reduceByKey(get_unique_dec).
	  map(lambda k, v: len(v) == 1).
	  collect()
# tabela jest spojna jesli dla kazdej kombinacji atrybutow
# mamy jedna decyzje
is_consistent = all(one_dec)
\end{lstlisting}
W tym algorytmie wykorzystujemy wcześniej omówioną metodę \textbf{reduceByKey} \ref{reduceByKey}. Funkcją redukującą jest tu scalenie
złączonych krotek $(a_1(u), ..., a_n(u), d)$ atrybutów i decyzji. Jeśli któraś z otrzymanych list jest większa niż jeden, znaczy to, że 
dla takiej samej krotki wartości na atrybutach $(a_1(u), ..., a_n(u))$ mieliśmy różną wartość decyzji $d$, więc tabela jest niespójna.
\section{Opis algorytmów}
Wszystkie z poniższych algorytmów są zrównoleglane ze względu na kolumnę, tak 
jak było to napisane we wstępie. Różnią się one wyborem metody \textbf{discretizeColumn}.
\subsection{Sortowanie}
Wszystkie niżej wymienione algorytmy operują na posortowanych kolumnach. By posortować całą tabelę najpierw
sprowadzam ją do postaci EAV, tj listy trójek (Obiekt, Atrybut, Wartość), a następnie sortuję tę listę 
najpierw względem wartości, potem obiektów, a następnie atrybutów, otrzymująć listę posortowanych wartości
zgrupowanych względem atrybutów. W Sparku wykonujemy sortowanie równoległe, za pomocą poniższego algorytmu,
będącego uogólnieniem mergesort:
\begin{itemize}
 \item dzielę listę na \textbf{numChunks} kawałków
 \item sortuję każdy kawałek niezależnie
 \item scalam posortowane kawałki
\end{itemize}
Kod algorytmu w języku python:
\begin{lstlisting}
# funkcja porownujaca
# wzgledem atrybutu, potem wartosci, a nastepnie nazwy obiektu
def compare(iterator):
  yield sorted(iterator, key=lambda x: (x[1], x[2], x[0]))

def merge_sort():
  numChunks = 10
  eav_rdd_part = Configuration.sc.
		 parallelize(eav, numChunks)
  self.eav =  eav_rdd_part.
	      mapPartitions(compare).
	      reduce(lambda x, y: 
	      sorted(x+y, key=lambda x: (x[1], x[2], x[0])))
\end{lstlisting}

Dla podziału na $k$ kawałków algorytm ma złożoność czasową 
$O(\underbrace{\frac{n}{k} \cdot log(\frac{n}{k})}_{\text{ dzielenie }} + \underbrace{\frac{n}{k} \cdot k)}_{\text{ scalanie }}$,
	    czyli $O(\frac{n}{k} \cdot log(\frac{n}{k}) + n)$
ma więc gdy $k \sim O(n)$ otrzymujemy algorytm liniowy.
	      
	      

\subsection{Algorytm bazowy}
Jest to prosty algorytm polegający na wykonywania cięć w równych odstępach, co $m$ obiektów, gdzie
$m$ jest parametrem.
Jego zaletą jest prostota, jednak w tym przypadku nie bierzemy pod uwagę tego jak tworzone podziały odnoszą się do decyzji.
Kod algorytmu:
\begin{lstlisting} 
def discretizeColumn(column, m):
  for ind, row in enumerate(column):
    yield (row[0], row[1], ind / m)
\end{lstlisting}

\begin{przyklad}[Przykładowa dyskretyzacja]
Dla tabeli:
\begin{center}
 \begin{tabular}{l || P{2.5cm} P{2.5cm} P{2.5cm} | P{1.5cm}}
  $U$     & Temperatura & Wilgotność & Zachmurzenie & Gra \\ 
  \hline
  $u_{1}$ & 0      & 30      & słabe          & Nie \\
  $u_{2}$ & 10     & 90      & umiarkowane    & Nie \\
  $u_{3}$ & 20     & 30      & słabe   	      & Tak \\
  $u_{4}$ & 15     & 50      & deszcz         & Nie \\
  $u_{5}$ & 15     & 30      & deszcz         & Nie \\
  $u_{6}$ & 30     & 70      & deszcz         & Nie \\
  $u_{7}$ & 15     & 30      & deszcz         & Nie \\
 \end{tabular}
\end{center}
Oraz $m$ = 3, mamy następującą dyskretyzację (dla kolumn Temperatura i Wilgotność):
\begin{center}
 \begin{tabular}{l || P{2.5cm} P{2.5cm} P{2.5cm} | P{1.5cm}}
  $U$     & Temperatura & Wilgotność & Zachmurzenie & Gra \\ 
  \hline
  $u_{1}$ & 0     & 0      & słabe          & Nie \\
  $u_{2}$ & 0     & 2      & umiarkowane    & Nie \\
  $u_{3}$ & 1     & 0      & słabe   	    & Tak \\
  $u_{4}$ & 0     & 1      & deszcz         & Nie \\
  $u_{5}$ & 1     & 0      & deszcz         & Nie \\
  $u_{6}$ & 2     & 1      & deszcz         & Nie \\
  $u_{7}$ & 1     & 1      & deszcz         & Nie \\
 \end{tabular}
\end{center}
\end{przyklad}
\subsection{OneR Discretizer}
W tym algorytmie wykonujemy cięcie w momencie, gdy częstość występowania
jednej z decyzji jest większa niż innych, przy czym dany przedział musi zawierać więcej niżej
$m$ elementów, gdzie $m$ jest ustalonym parametrem.
Algorytm:
\begin{lstlisting}
def discretize_column(column, m):
        new_dec = 0
        block_elements = 0
        dec_histogram_block = Counter()
        for elem in column:

            if block_elements > m and
	       most_common(dec_histogram_block) > block_elements / 2:
                block_elements = 0
                new_dec += 1
                dec_histogram_block = Counter()
            dec_histogram_block[decision(elem)] += 1
            block_elements += 1

            yield (elem[0], elem[1], new_dec)
\end{lstlisting}
\begin{przyklad}[Przykładowa dyskretyzacja]
Dla tabeli:
\begin{center}
 \begin{tabular}{l || P{2.5cm} P{2.5cm} P{2.5cm} | P{1.5cm}}
  $U$     & Temperatura & Wilgotność & Zachmurzenie & Gra \\ 
  \hline
  $u_{1}$ & 0      & 30      & słabe          & Nie \\
  $u_{2}$ & 10     & 90      & umiarkowane    & Nie \\
  $u_{3}$ & 20     & 30      & słabe   	      & Tak \\
  $u_{4}$ & 15     & 50      & deszcz         & Nie \\
  $u_{5}$ & 15     & 30      & deszcz         & Tak \\
  $u_{6}$ & 30     & 70      & deszcz         & Nie \\
  $u_{7}$ & 15     & 30      & deszcz         & Nie \\
 \end{tabular}
\end{center}
Oraz $m$ = 4, mamy następującą dyskretyzację (dla kolumn Temperatura i Wilgotność):
\begin{center}
 \begin{tabular}{l || P{2.5cm} P{2.5cm} P{2.5cm} | P{1.5cm}}
  $U$     & Temperatura & Wilgotność & Zachmurzenie & Gra \\ 
  \hline
  $u_{1}$ & 0     & 0      & słabe          & Nie \\
  $u_{2}$ & 0     & 1      & umiarkowane    & Nie \\
  $u_{3}$ & 1     & 0      & słabe   	    & Tak \\
  $u_{4}$ & 0     & 0      & deszcz         & Nie \\
  $u_{5}$ & 0     & 0      & deszcz         & Tak \\
  $u_{6}$ & 1     & 1      & deszcz         & Nie \\
  $u_{7}$ & 1     & 0      & deszcz         & Nie \\
 \end{tabular}
\end{center}
\end{przyklad}
\subsection{Dyskretyzacja za pomocą entropii}
Metoda została zaczerpnięta z \cite{Discretization}.
Entropię dla zbioru obiektów $U$ definiujemy jako:
\begin{align*}
  Ent(U) = - \sum_{d \in dec(U)} p_d \cdot log(p_d),
\end{align*}
Gdzie $dec(U)$ to zbiór decyzji obiektów ze zbioru $U$, oraz $p_d$ jest 
empirycznym prawdopodobieństwem decyzji $d$, liczonym z wzoru:
\begin{align*}
  p_d = \frac{|\{u \in U: dec(u) = d\}|}{|U|}.
\end{align*}
Entropia jest miarą informacji jaką mamy o danym zbiorze obiektów.
Jest ona zerowa, gdy w danym zbiorze obiektów mamy jedną decyzję, a więc 
mamy pełną informację, zaś największa gdy częstość wszystkich decyzji jest taka sama. 
Entropię dla danego cięcia $c$ na danym atrybucie $a$ możemy zdefiniować jako:
\begin{align*}
  E(a, c, U) = \frac{U_1}{n}Ent(U_1) + \frac{U_2}{n}Ent(U_2),
\end{align*}
gdzie $n$ jest liczbą obiektów w $U$ oraz $U_1$ i $U_2$ powstają z $U$ na podstawie cięcia $c$, w taki sposób, że
do $U_1$ należą obiekty o wartościach mniejszych od $c$ na atrybucie $a$ zaś $U_2 = U \setminus U_1$.
Dla danego cięcia możemy więc zdefniować miarę przyrostu informacji:
\begin{align*}
 Gain(a, c, U) = Ent(U) - E(a, c, U).
\end{align*}
Wybieramy takie cięcie, które maksymalizuje zysk informacji. Podział obiektów wykonujemy rekurencyjnie,
dopóki nie spełniony zostanie pewien warunek stopu, tj gdy zysk informacji, który osiągamy
dla danego optymalnego cięcia jest mniejszy niż zadana wartość progowa.
\subsubsection{Szukanie optymalnego cięcia - algorytm liniowy}
Możemy łatwo znależć optymalne cięcie dla danego zbioru obiektów w czasie $O(n^2)$ (gdzie $n$ jest
liczbą obiektów) za pomocą algorytmu (gdzie Ent(U) jest funkcją obliczającą entropię obiektów ze zbioru U):
\begin{lstlisting}
min_cut = 1
# mozliwe ciecia - pomiedzy obiektami od 1 do n
possible_cuts = range(1, n)
# ustawiamy wartosc minimalnego ciecia na nieskonczonosc
min_cut_value = INFINITY
for cut in possible_cuts:
  # Ent(U_k) liczymy w czasie liniowym ze wzgledu na długość U_k
  cut_value = (cut / n) * Ent(U[cut:]) + 
	      ((n - cut) / n) * Ent(U[:cut])
  if cut_value < min_cut_value:
    min_cut_value = cut_value
    min_cut = cut	      
\end{lstlisting}
Możemy jednak nie obliczać za każdym razem entropii, lecz aktualizować jej wartość na podstawie 
wcześniejszych wyników w czasie stałym. Prowadzi to do algorytmu liniowego ze względu na liczbę 
obiektów. Algorytm opiera się na obserwacji (zakładamy, że dla obiektów $u_1, ..., u_{k-1}$ mamy już obliczone 
częstości poszczególnych decyzji (słownik \textbf{decHist} w którym trzymamy histogram decyzji, za pomocą którego w czasie stałym
liczymy $p_d$), $Ent(U_k)$ oznacza entropię obiektów $u_1, ..., u_k$):
\begin{enumerate}
 \item $Ent(U_k) = Ent(U_{k-1}) + log(p_{dec(u_k)}) \cdot p_{dec(u_k)}$
 \item decHist[dec($u_k$)] += 1
 \item $Ent(U_k) = Ent(U_{k-1}) - log(p_{dec(u_k)}) \cdot p_{dec(u_k)}$
\end{enumerate}
Analogicznie trzymamy histogram dla pozostałych obiektów $u_k, ..., u_n$ aktualizując entropię drugiej połowy.
Otrzymujemy więc następujący algorytm liniowy:
\begin{lstlisting}
def find_cut(U):
  # decHist1 -> histogram dla 1 polowy
  # decHist2 -> histogram dla 2 polowy
  # p1 -> prawdopodobieństwo liczone na podstawie decHist1
  # p2 -> prawdopodobieństwo liczone na podstawie decHist2
  def updateEnt(U, k):
    Ent[U[k:]] = Ent([[k-1:]] - 
		 (-log(p1[dec(u_k)])*p1[dec(u_k)])
    Ent[U[:k]] = Ent[U[:k+1]] - 
		 (-log(p2[dec(u_k)])*p2[dec(u_k)])
    decHist1[dec(U[k])] += 1    
    decHist2[dec(U[k])] -= 1
    Ent[U[k:]] = Ent[U[k-1:]] - 
                 (-log(p1[dec(u_k)])*p1[dec(u_k)])
    Ent[U[:k]] = Ent[U[:k+1]] - 
                 (-log(p2[dec(u_k)])*p2[dec(u_k)])
    return Ent[U[k:]], Ent[U[:k]]
    
  min_cut = 1
  # mozliwe ciecia - pomiedzy obiektami od 1 do n
  possible_cuts = range(1, n)
  # ustawiamy wartosc minimalnego ciecia na nieskonczonosc
  min_cut_value = INFINITY
  Ent[0] = 0
  for cut in possible_cuts:
    # Ent(U_k) liczymy w czasie stalym gdy mamy juz wczesniej obliczone Ent(U_k-1)
    Ent1, Ent2 = updateEnt(U, cut)
    cut_value = (cut / n) * Ent1 + ((n - cut) / n) * Ent2
    if cut_value < min_cut_value:
      min_cut_value = cut_value
      min_cut = cut	
  # jesli nie jest spelnione kryterium stopu zwracamy minimalne ciecie
  # oraz ciecia dla odpowiednich podzbiorow obiektow
  if not stop_criterion_satisfied(U, min_cut_value):
    yield min_cut
    
    for cut in find_cut(U[k:]):
      yield cut 
    for cut in find_cut(U[:k]):
      yield cut
\end{lstlisting}
Koncowy Algorytm:
\begin{lstlisting}
# zwracamy tabele z odpowiednimi wartosciami disc_value
def discretizeColumn(column):
  disc_value = 0
  cuts_set = sorted(find_cut_lin(column))
  cur_cut = cuts_set[0]

  for i, elem in enumerate(column):
    if cur_cut == i:
      disc_value += 1
    if len(cuts_set) > disc_value:
      cur_cut = cuts_set[disc_value]

    yield (elem[0], elem[1], disc_value)
\end{lstlisting}
\begin{przyklad}[Przykładowa dyskretyzacja]
Dla tabeli:
\begin{center}
 \begin{tabular}{l || P{2.5cm} P{2.5cm} P{2.5cm} | P{1.5cm}}
  $U$     & Temperatura & Wilgotność & Zachmurzenie & Gra \\ 
  \hline
  $u_{1}$ & 0      & 30      & słabe          & Tak \\
  $u_{2}$ & 10     & 90      & umiarkowane    & Nie \\
  $u_{3}$ & 20     & 30      & słabe   	      & Nie \\
  $u_{4}$ & 15     & 50      & deszcz         & Tak \\
  $u_{5}$ & 15     & 30      & deszcz         & Nie \\
  $u_{6}$ & 30     & 70      & deszcz         & Nie \\
  $u_{7}$ & 15     & 30      & deszcz         & Nie \\
 \end{tabular}
\end{center}
mamy następującą dyskretyzację (dla kolumn Temperatura i Wilgotność):
\begin{center}
 \begin{tabular}{l || P{2.5cm} P{2.5cm} P{2.5cm} | P{1.5cm}}
  $U$     & Temperatura & Wilgotność & Zachmurzenie & Gra \\ 
  \hline
  $u_{1}$ & 0     & 0      & słabe          & Tak \\
  $u_{2}$ & 0     & 2      & umiarkowane    & Nie \\
  $u_{3}$ & 1     & 1      & słabe   	    & Nie \\
  $u_{4}$ & 0     & 2      & deszcz         & Tak \\
  $u_{5}$ & 1     & 1      & deszcz         & Nie \\
  $u_{6}$ & 1     & 2      & deszcz         & Nie \\
  $u_{7}$ & 1     & 1      & deszcz         & Nie \\
 \end{tabular}
\end{center}
\end{przyklad}

\section{Wyniki eksperymentów}
W przeprowadzonych eksperymentach wykorzystałem komputery z laboratorium MIMUW. Obliczenia wykonywałem 
na maksymalnie 48 komputerach o następujących parametrach:
  \begin{itemize}
   \item \textbf{pamięć RAM}: 32 komputery 4 GB RAM, 16 komputerów 8 GB RAM,
   \item \textbf{liczba jednostek CPU}: 8,
   \item \textbf{taktowanie pojedyńczego CPU}: 16 komputerów 3.1 GHz, 16 komputerów 3.4 GHz, 16 komputerów 2.8 GHz.
  \end{itemize}
Do eksperymentów wykorzystałem zbiór danych WAVEFORM z UCI Machine Learning Repository 
zawierający informacje na temat parametrów fal dźwiękowych oraz zbiór danych digits z pythonowej biblioteki sklearn (\cite{SklearnDocumentation})
zawierający opisy obrazków z narysowanymi odręcznie cyframi.
Pierwszy zbiór zawiera 5000 obiektów i 40 atrybutów oraz 3 wartości decyzyjne - rodzaj fali.
Drugi zawierał 1797 obiektów, 64 atrybuty (numer piksela na obrazku 8x8) oraz 10 wartości decyzyjnych - cyfrę.
Przeanalizowałem wyniki predykcji dla zbioru danych WAVEFORM w zależności od parametrów:
  \begin{itemize}
   \item dla metody simple jest to liczba obiektów dla pojedyńczego cięcia,
   \item dla metody OneR jest to minimalna liczba obiektów dla pojedyńczego cięcia.
  \end{itemize}

Predykcja została wykonana za pomocą drzewa decyzyjnego z biblioteki sklearn za pomocą 10-krotnej
kroswalidacji metodą accuracy.

\begin{figure}
 \caption{Accuracy score dla zbioru danych WAVEFORM dla wybranych metod dyskretyzacji w zależności od parametrów}\label{fig:disc_scores}
 \centering\includegraphics[scale=0.8]{./wykresy/disc_scores2.png}
\end{figure}

Z wykresu \ref{fig:disc_scores} widzimy, że wyniki dla dyskretyzacji są lepsze od baseline-u, 
zaś najlepsze dla metody prostej i parametru $m = 15$, lecz 
średnio dla analizowanych danych najlepsza okazała się metoda entropii.
\subsection{Skalowalność}
Zobaczmy jak wypadły eksperymenty pod kątem skalowalności. Wykonałem eksperymenty z następującą liczbą komputerów: 1, 5, 10, 20, 30, 40.
Zbadałem jak rozkładają się poszczególne metryki w zależności od liczby komputerów i użytej metody.

\begin{figure}
 \caption{Metryki skalowalności dla różnych liczb komputerów oraz wybranych metod dyskretyzacji}\label{fig:sc_disc}
 \centering\includegraphics[scale=0.8]{./wykresy/sc_disc2.png}
\end{figure}

Z wykresu \ref{fig:sc_disc} widzimy, że zarówno speed\_up jak i scale\_up wraz ze zwiększaniem liczby komputerów odbiega coraz bardziej od idealnej sytuacji 
(zaznaczonej na wykresie na brązowo). Jest to zgodne z prawem Gustafsona. Gdy liczba komputerów wzrasta, jest coraz większy narzut czasowy
ze względu na przesyłanie danych po sieci. Widzimy jednak, że gdy liczba komputerów zaczyna przekraczać 10, obie 
metryki zachowują się coraz gorzej, przyrost jest coraz wolniejszy lub zerowy. Do 10 jednostek algorytmy skalują się więc dobrze, 
zaś stosowanie większej liczby komputerów jest już nieopłacalne.



\chapter{Generowanie nowych cech na podstawie algorytmów genetycznych}

W tym rozdziale omówię uogólnienie pojęcia dyskretyzacji prowadzące do generowania nowych cech.
Przedstawię wyznaczanie nowych cech za pomocą hiperpłaszczyzn i algorytmu genetycznego. Zaczerpnięty algorytm
pochodzi z pracy \cite{Hyperplanes}.

\section{Dyskretyzacja jako generowanie nowych cech oraz generowanie nowych cech przez hiperpłaszczyzny}
Na dyskretyzację, a konkretnie na cięcie na pewnym atrybucie rzeczywistym możemy też patrzeć
jak na nową cechę zdefiniowaną przez funkcję charakterystyczną wyznaczonej przez cięcie półprzestrzeni. Cięcie o wartości
$c$ na atrybucie $a_k$ możemy traktować jako cechę zdefiniowaną następująco:
\begin{align*}
v_{c,k}(u) = \begin{cases} 1 \text{ dla } a_k(u) \leq c \\ 0 \text{ dla } a_k(u) > c. \end{cases}
\end{align*}
Cięcie na danym atrybucie definiuje nam jedynie hiperpłaszczyznę równoległą do odpowiednich osi. 
Możemy zdefiniować nowe cechy jako funkcje charakterystyczne dowolnych półprzestrzeni:
\begin{align*}
v_{w_1, w_2, ..., w_n, c}(u) = \begin{cases} 1 \text{ dla } w_1 \cdot a_1(u) + w_2 \cdot a_2(u) + ... + w_n \cdot a_n(u) \leq c 
\\ 0 \text{ dla } w_1 \cdot a_1(u) + w_2 \cdot a_2(u) + ... + w_n \cdot a_n(u) > c. \end{cases}
\end{align*}
Ogólne półprzestrzenie mają o wiele większą moc wyrazu i zwykle wystarczy ich mniej by znależć minimalny zbiór cięć dla
danej przestrzeni, tj tak podzielić przestrzeń, aby otrzymać spójną tabelę.
Jednak przestrzeń poszukiwań odpowiednich parametrów, tj wag $w_k$ jest nieskończona. Dla danego zbioru wag, tj dla danej półprzestrzeni,
możemy jednak ocenić jak dobre jest dane cięcie na podstawie funkcji oceny. Jedną z popularniejszych miar jest miara rozróżnialności. 
Zdefiniujmy $C_i^R$ oraz $C_i^L$ jako liczbę obiektów o decyzji $i$ należących do odpowiednich półprzestrzeni:
\begin{align*}
C_i^R(H) = |\{u \in U: dec(u) = i \wedge w_1 \cdot a_1(u) + w_2 \cdot a_2(u) + ... + w_n \cdot a_n(u) > c\}| \\
C_i^L(H) = |\{u \in U: dec(u) = i \wedge w_1 \cdot a_1(u) + w_2 \cdot a_2(u) + ... + w_n \cdot a_n(u) \leq c\}|
\end{align*}
Na podstawie tych wielkości definiujemy miarę rozróżnialności dla hiperpłaszczyzny $H$:
\begin{align*}
discMeasure(H) = \sum_{i \neq j} C_i^R(h) \cdot C_j^L(H)
\end{align*}
Mając funkcję oceny możemy więc z powodzeniem stosować pewne heurystyki doboru wag pozwalające wyznaczyć
niezbędne punkty za pomocą algorytmów genetycznych. Konkretny algorytm omówię w następnym
rozdziale.
\section{Algorytmy ewolucyjne}
Algorytmy ewolucyjne \cite{Genetic} są jedną popularniejszych metod heurystycznych metod optymalizacji, jak np. symulowane wyżarzanie. Sprawdzają się,
gdy mamy dobrze określoną funkcję celu oraz potrafimy odpowiednio zamodelować nasze uniwersum. Algorytm genetyczny polega na iteracyjnym 
poprawianiu wyniku poprzez dobieranie do następnej fazy tych osobników, które osiągają najlepsze wartośc funkcji celu. Podstawowym
pojęciem jest tu populacji i osobnik. Osobnik jest pewnym obiektem matematycznym, najczęściej wektorem zerojedynkowym, który modeluje
wielkości, które chcemy optymalizować. W przypadku problemu szukania optymalnej hiperpłaszczyzny będzie to wektor reprezentujący wagi $w_i$.
Populacja zaś jest zbiorem osobników. Algorytm genetyczny można podzielić na następujące etapy:
\begin{enumerate}
 \item Inicjowanie populacji początkowej
 \begin{enumerate}
  \item Inicjowanie losowe
  \item Inicjowanie wg wcześniej ustalonego schematu
 \end{enumerate}
 \item W pętli, aż do spełnienia pewnego warunku stopu wykonuj:
 \begin{enumerate}
  \item Liczymy funkcję oceny każdego osobnika w populacji. Najlepsze osobniki poddajemy reprodukcji.
  \item Proces repropdukcji najlepszych osobników:
  \begin{enumerate}
   \item Krzyżowanie - łączenie genotypów rodziców
   \item Mutacja - wprowadzanie losowych zmian
  \end{enumerate}
  \item Po procesie reprodukcji wybieramy najlepsze osobniki.
  \item Jeśli został spełniony warunek stopu wybieramy i zwracamy najlepszego osobnika w populacji.
 \end{enumerate}
\end{enumerate}
Podstawowymi zadaniami przy projektowaniu algorytmu genetycznego są:
\begin{itemize}
 \item ustalenie genomu osobnika jako reprezentanta wyniku
 \item ustalenie funkcji oceny
 \item dobór parametrów (warunek stopu, prawdopodobieństwo krzyżowania, mutacji, rozmiar populacji)
\end{itemize}
Dla problemu znajdowania optymalnej hiperpłaszczyzny funkcją oceny będzie wspomniana wcześniej miara \textbf{discMeasure}.
Pozostaje ustalić genom osobnika (hiperpłaszczyzny). Będziemy reprezentować hiperpłaszczyznę w przestrzeni $\mathbb{R}^n$
jako zbiór $n-1$ liniowo niezależnych wektorów i przesunięcia. 
\subsection{Reprezentacja hiperpłaszczyzny}
Ustalmy jedną z osi, niech to będzie $x_1$ odpowiadająca atrybutowi $a_1$ i liczbę naturalną $b$, będzie to parametr odpowiadający
za zbiór, z którego inicjujemy populację początkową. Z każdej dwu-wymiarowej płaszczyzny $L(x_1, x_i)$ wybieramy wektory 
$v_1^i, v_2^i, ..., v_{2^b}^i$ (nierównoległe do $x_1$) zdefiniowane przez:
\begin{align*}
v_j^i = [\alpha_j^i, 0, ..., 0, \underbrace{1}_{\text{i-ta pozycja}} , 0, ..., 0] \text{ dla } i=2, ..., n \text{ oraz } j=1, ..., 2^b 
\end{align*}
Wektory te będziemy inicjować przez losowy dobór $2^b$ liczb: $\alpha_1^i, \alpha_2^i, ..., \alpha_{2^b}^i$.
Każdy z wektorów $v_j^i$ może być reprezentowany przez $b$ bitów odpowiadających reprezentacji
binarnej liczby $j$. Rozważmy zbiór $n-1$ wektorów $\{v_{j_2}^2, v_{j_3}^3, ..., v_{j_n}^n\}$:
\begin{align*}
 v_{j_2}^2 = [\alpha_{j_2}^2, 1, 0, 0,  ..., 0] \\
 v_{j_3}^3 = [\alpha_{j_3}^3, 0, 1, 0,  ..., 0] \\
 ..... \\
 v_{j_n}^n = [\alpha_{j_n}^n, 0, 0, 0,  ..., 1]
\end{align*}
Łatwo widać, że wektory dla dowolnego doboru indeksów $j_2, j_3, ..., j_n$ wektory $\{v_{j_2}^2, v_{j_3}^3, ..., v_{j_n}^n\}$ 
są liniowo niezależne. Wybierając przesunięcie $P_1 = (p, 0, ..., 0)$ na osi $x_1$, możemy zdefiniować 
hiperpłaszczyznę $(P_1, v_{j_2}^2, v_{j_3}^3, ..., v_{j_n}^n)$ przez:
\begin{align*}
 H = \{(x_1, x_2, ..., x_n\} \in \mathbb{R}^n : x_1 - \alpha_{j_2}^2 x_2 - \alpha_{j_3}^3 x_3 - ... - \alpha_{j_n}^n x_n - p = 0\}
\end{align*}
Osobniki będziemy reprezentować przez $n-1$ liczb b-bitowych $\alpha_1, \alpha_2, ..., \alpha_{n-1}$, które będą 
reprezentować wektory (dla osi $x_1$):
\begin{align*}
 v_1 = [\alpha_1, 1, 0, 0,  ..., 0] \\
 v_2 = [\alpha_2, 0, 1, 0,  ..., 0] \\
 ..... \\
 v_{n-1} = [\alpha_{n-1}, 0, 0, 0,  ..., 1]
\end{align*}
Są one liniowo niezależne i rozpinają $n-1$ wymiarową podprzestrzeń liniową.
Taka reprezentacja umożliwia nam sprawdzenie w czasie $O(n)$ po której stronie hiperpłaszczyzny znajduje się dany obiekt $u \in U$ oraz 
policzyć w czasie $O(n)$ rzut wektora $u$ na $x_1$ równoległy do $L = lin(v_{j_2}^2, v_{j_3}^3, ..., v_{j_n}^n)$. Odpowiednie funkcje, $Test(u)$
licząca, po której stronie hiperpłaszczyzny jest obiekt $u$:
\begin{align*}
 Test(u) = \begin{cases} 1 \text{ jeśli } a_1(u) - \alpha_{j_2}^2 a_2(u) - \alpha_{j_3}^3 a_3(u) - 
 ... - \alpha_{j_n}^n a_n(u) \geq p \\ 0 \text{ wpp } \end{cases}
\end{align*}
Oraz $Projection(u)$ licząca rzut obiektu $u$ na odpowiednią oś (tu $x_1$):
\begin{align*}
 Projection(u) = a_1(u) - \alpha_{j_2}^2 a_2(u) - \alpha_{j_3}^3 a_3(u) - ... - \alpha_{j_n}^n a_n(u)
\end{align*}
\subsection{Szukanie optymalnej hiperpłaszczyzny za pomocą algorytmu genetycznego}
Za pomocą algorytmu genetycznego będziemy chcieli szukać optymalnej hiperłpaszczyzny dla danej osi. Dla każdego osobnika liczymy rzut obiektów na 
odpowiednią oś za pomocą funkcji \textbf{Projection}, a następnie wybieramy najlepszą hiperpłaszczyznę równoległą do podprzestrzeni liniowej 
reprezentowanej przez osobnika za pomocą funkcji oceny. Na rysunku \ref{fig:hiperp} widzimy przykład dla przestrzeni dwuwymiarowej.
\begin{figure}
 \caption{Interpretacja szukania optymalnej hiperpłaszczyzny w przestrzeni dwuwymiarowej}\label{fig:hiperp}
 \includegraphics[scale=0.8]{hiperp.png}
\end{figure}
Powtarzając algorytm dla każdej z osi,
wybierzemy najlepszą hiperpłaszczyznę. Hiperpłaszczyzny szukamy do momentu, w którym tabela otrzymana z nowych cech indykatorów odpowiednich 
półprzestrzeni będzie spójna, bądź spełniony zostanie warunek stopu, o którym opowiem w następnych rozdziałach. Na rysunku \ref{fig:obszary}
widać przykładowy podział obiektów w przestrzeni dwuwymiarowej po znalezieniu dwóch hiperpłaszczyzn.
\begin{figure}
 \caption{Przykładowy podział obiektów przez hiperpłaszczyzny. Na rysunku widzimy podział na 3 obszary, 2 z nich zawierają obiekty
 o różnych decyzjach (różne kształy oznaczają różne decyzje).}\label{fig:obszary}
 \includegraphics[scale=0.8]{./wykresy/obszary.png}
\end{figure}
Po znalezieniu kolejnych hiperpłaszczyzn grupy obszarów niespójnych będą się stale zmniejszać. Funkcję liczę jedynie dla obszarów
niespójnych, które nie są jeszcze podzielone. Funkcję celu dla listy obszarów niespójnych wyliczam z wzoru (kod w języku python):
\begin{lstlisting}
# unconsistent_groups jest lista niespojnych grup obiektow
award = sum([discMeasure(H, U) for U in unconsistent_groups])
\end{lstlisting}
Pełny kod algorytmu w języku python znajduje sie tez w załączniku \ref{App:hyperp}:
    
\subsection{Analiza złożoności czasowej algorytmu}
W skócie algorytm można przedstawić za pomocą pseudokodu: \\

\begin{algorithm}[H]
 \KwData{Tabela decyzyjna $\mathbb{S}$ \\
         Zbiór obiektów $\mathbb{U}$ \\
	 Zbiór atrybutów $\mathbb{A}$ \\
         Lista niespójnych grup obiektów $UncGroups$ \\
         Nowa tabela $\mathbb{B}$}
 \KwResult{Nowa tabela $\mathbb{B}$ z nowymi cechami}
  $ B = \emptyset$\;
  $ UncGroups = \mathbb{U}$\;
  $ H = \emptyset$\;
 \While{$UncGroups$ $\neg$ $\emptyset$}{
  \For{$atr$ in $\mathbb{A}$}{
      $H$ = $H$ $\cup$ najlepsza hiperpłaszczyzna znaleziona za pomocą algorytmu genetycznego 
            dla osi odpowiadającej atrybutowi $atr$ i grupom obiektów z $UncGroups$
  }
 }
 $OptH$ = najlepsza hiperpłaszczyzna z zbioru $H$. \\
 $B$ = $B$ $\cup$ cecha wyznaczona przez $OptH$. \\
 Aktualizuj $UncGroups$ na podstawie $B$
\end{algorithm} 

Koszt czasowy zewnętrznej pętli while ciężko jest oszacować, gdyż płaszczyzna wybierana jest w sposób niedeterministyczny. Możemy ustalić pewien 
próg czasowy, w którym algorytm musi się zatrzymać jeśli $UncGroups$ wciąż jest niepuste. Koszt czasowy pętli wewnętrznej wynosi
$O(n \cdot gen)$, gdzie $gen$ to koszt czasowy algorytmu genetycznego. Przyjrzyjmy się pseudokodowi algorytmu genetycznego oraz
wyznaczania funkcji oceny: \\
\begin{algorithm}[H]
 \KwData{Populacja $P$}
 {maksymalna liczba iteracji $max\_iter$}
  $ i = 0 $\;
  $ P = init\_generation $\;
 \While{$i \leq max\_iter$}{
  i = i+1 \\
  \For{$Osobnik \in P$}{
  $Oceny$ = $Oceny$ $\cup$ wartość oceny dla $Osobnik$
  }
 Wybierz najlepszych osobników. \\
 Skrzyżuj i mutuj najlepsze osobniki. \\
 $P$ = Zainicjuj nowe pokolenie. 
 }
 zwróć najlepszego osobnika. \
 \caption{przeszukiwanie genetyczne}
\end{algorithm}

\begin{algorithm}
 \KwData{Osobnik $o$}
	{Niespójne grupy obiektow $UncGroups$} \\
  rzuty = oblicz rzuty obiektów z $UncGroups$ na odpowiednią oś równolegle do podprzestrzeni wyznaczonej przez $o$.\\
  sortuj rzuty. \\
  oceny = licz wartośc $DiscMeasure$ dla wszystkich rzutów. \\
  zwróć najlepszą ocenę. \
  \caption{wyznaczanie funkcji oceny}
\end{algorithm}

Złożonośc czasowa wyznaczania rzutu dla obiektu wynosi $O(n)$, zaś liczenia funkcji oceny dla wszystkich rzutów $O(k)$, gdzie $k$ jest liczbą 
obiektów (dla posortowanych obiektów aktualizujemy w czasie stałym wartość \textbf{DiscMeasure}). A więc koszt czasowy wyznaczania funkcji oceny wynosi
$O(n \cdot k + k \cdot log(k) + k) = O(k(log(k) + n))$. Koszt czasowy algorytmu genetycznego wynosi więc 
$O(max\_iter \cdot |P| \cdot k(log(k) + n))$. A więc koszt szukania pojedyńczej hiperplaszczyzny wynosi 
$O(n \cdot max\_iter \cdot |P| \cdot k(log(k) + n))$ i zależy od doboru parametrów $|P|$ i $max\_iter$. W idealnym modelu obliczeń 
   równoległych, gdy mamy do dyspozycji $K$ komputerów koszt czasowy wyniesie:
$O(\frac{n}{K} \cdot max\_iter \cdot \frac{|P|}{K} \cdot k(log(k) + n))$, a więc dla dostatecznie dużych $K$ dostajemy:
$O(max\_iter \cdot k(log(k) + n))$.
   
\section{Indukcja drzewa decyzyjnego na podstawie cięć przez hiperpłaszczyzny}
Optymalne hiperpłaszczyzny mogą być użyte nie tylko ekstracji nowych cech, ale również do indukcji drzewa
decyzyjnego. W standardowym drzewie, w węźle mamy funkcję $test$ odpowiadającą atrybutom, dla wartości ciągłych zwykle
jest to:
\begin{align*}
test(x) = \begin{cases} 1 \text{ jeśli } a_i(x) > c \\ 0 \text{ wpp } \end{cases}
\end{align*}
Możemy jednak użyć funkcji testu opartej o hiperplaszczyzny, wtedy drzewo będzie miało zwykle mniejszą głębokośc
(hiperpłaszczyzny mogą mieć wyższą wartośc funkcji oceny na danym węźle), ale większy jest koszt czasowy predykcji
(funkcja testu działa w czasie $O(n)$ gdzie $n$ jest liczbą atrybutów) oraz szukania drzewa. Funkcja testu dla 
uogólnionego drzewa (dla wag $w_i$):
  \begin{align*}
test(x) = \begin{cases} 1 \text{ jeśli } w_1 a_1(x) + ... + w_n a_n(x) > c \\ 0 \text{ wpp } \end{cases}
\end{align*}
Oraz funkcja dla testu dla reprezentacji hiperplaszczyzny z algorytmu genetycznego (gdy $x_1$ jest najlepszą osią):
\begin{align*}
 test(x) = \begin{cases} 1 \text{ jeśli } a_1(x) - \alpha_{j_2}^2 a_2(x) - \alpha_{j_3}^3 a_3(x) - 
 ... - \alpha_{j_n}^n a_n(x) \geq p \\ 0 \text{ wpp } \end{cases}
\end{align*}
Algorytm znajdowania drzewa przez algorytm genetyczny w języku python:
\begin{lstlisting}
def count_decision_tree(objects, sc=None):
# jesli mozemy podjac decyzje dla danego zbioru obiektow
# zwracamy koncowy wezel
        decision = dpoints_strategy.decision(objects)
        if decision is not None:
            return DecisionTree(decision, 0, 0)

# szukamy najlepszej hiperplaszczyzny algorytmem genetycznym
# (tak jak dla ekstrakcji cech)
        best_hyperplane = search_best_hyperplane([objects], sc)
        hyperplane_indicator = count_objects_positions(
			       best_hyperplane, objects)
        left_son_objects = map(
			   lambda x: x[1], 
			   filter(
			   lambda (i, x): 
			   hyperplane_indicator[i] == 0, 
			   enumerate(objects)))
        right_son_objects = map(
			    lambda x: x[1], 
			    filter(
			    lambda (i, x): 
			    hyperplane_indicator[i] == 1, 
			    enumerate(objects)))

        return DecisionTree(best_hyperplane, 
			    count_decision_tree(left_son_objects),
                            count_decision_tree(right_son_objects))
\end{lstlisting}

\subsection{Generowanie lasu losowego}
Jako, że w algrytmie genetycznym początkowe wartości wektorów dobierane są w sposób losowy, 
drzewa decyzyjne generowane kilkukrotnie przez algorytm genetyczny mogą się od siebie różnić.
Przy jednokrotnym generowaniu drzewa istnieje prawdopodobieństwo utknięcia w lokalnym maksimum funkcji celu.
Możemy więc stworzyć wiele drzew i podjąć decyzję na podstawie głosowania, tworząc coś na kształt lasu losowego (\cite{RandomForest}).
Algorytm w języku python:
\begin{lstlisting}
# objects <- lista wszyskich obiektow w tabeli
def count_random_forest(size=50):	
	return [count_decision_tree(objects) for _ in range(size)]
      
def predict(object):
	forest = count_random_forest()
# liczymy predykcje dla kazdego drzewa
	decisions = map(lambda tree: tree.predict(object), forest)
# zwracamy najczesciej wystpujaca decyzje
	return most_common(Counter(decisions))		
\end{lstlisting}

\section{Drzewo SVM}
Dla celów analizy porównawczej możemy stworzyć drzewo, w którym funkcją testu
również będzie funkcja charakterystyczna półprzestrzeni, ale odpowiednią hiperpłaszczyznę
będziemy szukać nie za pomocą algorytmu genetycznego, ale algorytmu SVM (\cite{SVM}), szukającego 
hiperpłaszczyzny maksymalizującej sumę odległości od obiektów. Rysunek ilustruje działanie
algorytmu SVM.
\begin{figure}
 \caption{Podział obiektów przez hiperpłaszczyznę SVM}
 \includegraphics[scale=0.7]{svm_hiperp.png}
\end{figure}
Kod algorytmu w języku python:
\begin{lstlisting}
def count_decision_tree(objects, sc=None):
        decision = dpoints_strategy.decision(objects)
        if decision is not None:
            return DecisionTree(decision, 0, 0)

        X = [self.table[i] for i in objects]
        y = [self.dec[i] for i in objects]
        svm = LinearSVC()
# wyliczanie SVM z zewnetrznej biblioteki
        svm.fit(X, y)
# wybieranie wspolczynnikow optymalnej hiperplaszczyzny
        coefs = svm.coef_[0]
        intercept = svm.intercept_[0]
        best_hyperplane = (intercept, coefs)
        hyperplane_indicator = map(
			       lambda r: 
			       (np.dot(
			       list(r), coefs) + intercept > 0),
			       X)
        left_son_objects = map(
			   lambda x: x[1], 
			   filter(
			   lambda (i, x): 
			   not hyperplane_indicator[i], 
			   enumerate(objects)))
        right_son_objects = map(
		            lambda x: x[1], 
		            filter(lambda (i, x): 
		            hyperplane_indicator[i], 
		            enumerate(objects)))

        return DecisionTree(best_hyperplane, 
			    count_decision_tree(left_son_objects),
                            count_decision_tree(right_son_objects))
\end{lstlisting}

\section{Strategie klasyfikacji bliskich punktów o różnej decyzji oraz przycinanie}\label{sec:strategy}
Zarówno w algorytmie indukcji drzewa decyzyjnego jak i w algorytmie szukania 
nowych cech pojawiła się funkcja \textbf{dpoints\_strategy.decison(objects)}. W przypadku
gdy pojawią się punkty będące ``blisko'' siebie mające różne decyzje algorytmowi genetycznemu ciężko
będzie ``wcisnąć'' się pomiędzy te punkty przy losowej optymalizacji. Dlatego w przypadku indukcji drzewa 
wyliczamy decyzję dla takich punktów w inny, deterministyczny sposób, natomiast przy generowaniu cech
takie punkty ignorujemy. Pytanie jakie się nasuwają są dwa:
\begin{enumerate}
 \item Kiedy uznawać, że punkty są bliskie?
 \item W jaki sposób podejmować decyzję dla takich punktów?
\end{enumerate}
W swojej pracy uznaję punkty za bliskie, gdy średnica zbioru jest mniejsza niż pewien procent 
średnicy całego zbioru (problem pojawia się gdy występują outliery, wtedy można liczyć średnicę
bez outlierów). Decyzją zaś jest najczęściej występująca decyzja w danym zbiorze. Kod algorytmu w języku python:
\begin{lstlisting}
# max_dist <- srednia calego zbioru obiektow
def decision(objects, ratio):
	max_dist_objects = count_max_dist(objects)
	if max_dist_objects / max_dist < ratio:
	  return most_common(Counter(decisions(objects)))
	return None
\end{lstlisting}
Dodatkowo, nawet jeśli punkty nie będą blisko siebie i algorytm genetyczny wyszuka optymalne płaszczyzny może
wystąpić ryzyko nadmiernego dopasowania do danych i przeuczenia. Jedną z metod ustalania momentu stopu użytych w pracy 
jest odcięcie w momencie, gdy najliczniejsza decyzja stanowi więcej niż pewien ustalony odsetek. 
W eksperymentach dla generowania nowych cech użyłem właśnie tej strategii odcięcia.

\section{Wyniki eksperymentów}
Eksperymenty dla tej części również wykonywałem w laboratorium MIMUW przy użyciu takiego samego zestawu komputerów
oraz dla zestawu danych WAVEFORM używanym wcześniej przy dyskretyzacji.
\subsection{Generowanie nowych cech}
Przetestowałem skuteczność oraz czas trwania algorytmu genetycznego w zależności od parametrów, którymi są:
  \begin{itemize}
   \item rozmiar genomu
   \item rozmiar populacji
   \item maksymalna liczba iteracji
  \end{itemize}
Jako strategię klasyfikacji bliskich punktów o różnej decyzji wybrałem zatrzymanie przeszukiwania w momencie gdy mamy 
0.8 całości obiektów o jednej decyzji. Wygenerowane nowe cechy zostały dołączone do wyjściowej tabeli. Następnie zbadałem
wyniki predykcji drzewem decyzyjnym z biblioteki sklearn z zastosowaniem 10-krotnej kroswalidacji.
Wyniki można zobaczyć na wykresach \ref{fig:fe_score} oraz \ref{fig:fe_time}.

\begin{figure}
 \caption{Accuracy score dla zbioru danych WAVEFORM z nowymi cechami w zależności od parametrów algorytmu genetycznego}\label{fig:fe_score}
 \centering\includegraphics[scale=0.8]{./wykresy/fe_score2.png}
\end{figure}

\begin{figure}
 \caption{Czas generowania nowych cech dla zbioru danych WAVEFORM w zależności od parametrów algorytmu genetycznego}\label{fig:fe_time}
 \centering\includegraphics[scale=0.8]{./wykresy/fe_time2.png}
\end{figure}

Widzimy, że zwiększając liczbę populacji uzyskujemy lepsze wyniki, jednak odbywa się to kosztem
czasu przetwarzania. Jeśli chodzi o maksymalną liczbę iteracji, już przy 15 iteracjach algorytm zbiega do 
dobrego rozwiązania, zwiększając liczbe iteracji, uciekamy od rozwiązania optymalnego, zwiększając jeszcze 
czas przetwarzania. Również rozmiar genomu wpływa na czas przetwarzania, im większy tym szybciej
uzyskujemy rozwiązanie, a więc lepiej ustalić większy rozmiar genomu. Podsumowując wyniki eksperymentów, najlepsze
wyniki uzyskaliśmy dla dużych rozmiarów genomów - 20, małej liczby iteracji - 15 i dużego rozmiaru populacji - 50.

\subsection{Generowanie drzewa decyzyjnego}
Podobne eksperymenty przeprowadziłem dla generowania drzewa decyzyjnego przez cięcia hiperpłaszczyznami.
Ustaliłem rozmiar genomu na 35 i wprowadziłem parametr mówiący o tym, w którym momencie kończymy przeszukiwanie
dla danego węzła: gdy 0.8 obiektów ma taką samą decyzję lub gdy tych obiektów jest 0.9.
Poziomą kreską został zaznaczony wynik jaki uzyskałem drzewem decyzyjnym z biblioteki sklearn.
Wyniki można zaobserować na wykresach \ref{fig:tree_score} oraz \ref{fig:tree_time}.

  
\begin{figure}
 \caption{Accuracy score dla zbioru danych WAVEFORM dla drzewa decyzyjnego wygenerowanego za pomocą algorytmu genetycznego}\label{fig:tree_score}
 \centering\includegraphics[scale=0.8]{./wykresy/tree_score2.png}
\end{figure}

\begin{figure}
 \caption{Czas generowania drzewa decyzyjnego dla zbioru danych WAVEFORM w zależności od parametrów algorytmu genetycznego}\label{fig:tree_time}
 \centering\includegraphics[scale=0.8]{./wykresy/tree_time2.png}
\end{figure}

Widzimy, że zarówno wyniki jak i czas pogarszają się gdy zwiększymy parametr warunku stopu. Przy zbyt dużym parametrze, 
szukamy płaszczyzn dłużej, co może prowadzić do nadmiernego dopasowania do danych, i w efekcie przeuczenia.
W przeciwieństwie do algorytmu generowania nowych cech, tutaj zwiększając liczbę iteracji uzyskujemy lepsze wyniki,
zaś optymalnym rozmiarem populacji będzie w tym przypadku 30. W przypadku generowania drzewa mamy prostszą funkcję
oceny niż w przypadku generowania nowych cech, dlatego prawdopodobnie potrzebujemy mniejszego rozmiaru populacji.
Różnica jaką otrzymujemy w wynikach jest jednak niewielka, opłaca się więc, tak jak w przypadku generowania nowych cech
brać mniejsze rozmiary populacji i maksymalną liczbę iteracji, gdyż algorytm dość szybko zbiega do dobrego rozwiązania.

\subsection{skalowalność}
Tak samo jak dla dyskretyzacji zbadałem jak zachowują się metryki opisujące skalowalność dla szukania nowych cech i generowania
drzewa decyzyjnego przez hiperpłaszczyzny za pomocą algorytmu genetycznego. Tak samo jak wcześniej eksperymenty
przeprowadziłem dla: 1, 5, 10, 20, 30 i 40 komputerów. Wyniki możemy zaobserować na wykresie \ref{fig:sc_fe}.

\begin{figure}
 \caption{Wskaźniki metryki skalowalności dla wyszkukiwania nowych cech i indukcji drzewa decyzyjnego algorytmem genetycznym}\label{fig:sc_fe}
 \centering\includegraphics[scale=0.8]{./wykresy/sc_fe2.png}
\end{figure}

Podobnie jak dla dyskretyzacji widzimy, że powyżej 10 komputerów przyrosty są coraz mniejsze, jeszcze mniejsze niż w przypadku
dyskretyzacji. W przypadku wyszukiwania genetycznego warto więc przeprowadzać obliczenia na mocniejszych jednostkach obliczeniowych
rozpraszając na nie więcej niż 20 komputerów.
\newpage

\subsection{Porównanie wyników}
Oto porównanie najlepszych wyników jakie uzyskałem dla zbiorów danych WAVEFORM oraz digits z biblioteki sklearn
przy zastosowaniu 5-krotnej kroswalidacji.
Wyszukiwanie algorytmem genetycznym odbywało się na 20 komputerach. Parametry jakie ustawiłem:
\begin{itemize}
 \item maksymalna liczba iteracji: 20
 \item rozmiar genomu: 50
 \item rozmiar populacji: 35
 \item liczba drzew w lesie losowym: 10
\end{itemize}

\begin{table}
\caption{Wyniki dla zbioru danych WAVEFORM}\label{tab:res_wav}
 \begin{tabular}{l || P{2cm} | P{2cm} | P{2cm} | P{2cm} | P{2cm}}
  Metoda/parametr     & Drzewo decyzyjne & Drzewo decyzyjne SVM & Drzewo decyzyjne genetycznie & Drzewo decyzyjne + nowe cechy & Las losowy gentycznie\\ 
  \hline
  Score (accuracy) & 0.7576     & 0.7482     & 0.8231 & 0.793 & 0.841\\
  \hline
  Czas (s) & 0.012     & 0.41      & 53.04    & 191.00 & 183.09 \\
  \hline
  Odchylenie standardowe & 0.0097     & 0.012      & 0.018    & 0.019 & 0.011 \\
 \end{tabular}
\end{table}

\begin{table}
\caption{Wyniki dla zbioru danych digits}\label{tab:res_dig}
 \begin{tabular}{l || P{2cm} | P{2cm} | P{2cm} | P{2cm} | P{2cm}}
  Metoda/parametr     & Drzewo decyzyjne & Drzewo decyzyjne SVM & Drzewo decyzyjne genetycznie & Drzewo decyzyjne + nowe cechy & Las losowy gentycznie\\ 
  \hline
  Score (accuracy) & 0.869     & 0.971     & 0.915 & 0.941 & 0.947 \\
  \hline
  Czas (s) & 0.006     & 0.11      & 23.13    & 43.07 & 52.15 \\
  \hline
  Odchylenie standardowe & 0.013     & 0.010      & 0.024   & 0.022 & 0.013 \\
 \end{tabular}
\end{table}

Z tabel \ref{tab:res_wav} oraz \ref{tab:res_dig}
widzimy, że w obu przypadkach bardzo dobre wyniki uzyskujemy za pomocą drzewa wygenerowanego przez algorytm genetyczny 
oraz lasów losowych złożonych z tych drzew. 
Drzewa SVM dla digits również osiągnęły znakomity wynik, jednak dla WAVEFORM uzyskują zbliżone do zwykłego drzewa decyzyjnego z biblioteki sklearn. Dobre wynik uzyskaliśmy również dodając nowe cechy. 
Metody korzystające z cięć przez hiperplaszczyzny
uzyskują więc bardzo dobre wyniki, odbywa się to jednak kosztem złożności czasowej. Ponadto drzewa wygenerowane metodą SVM lub algorytmem genetycznym ze względu na cięcia hiperpłaszczyznami
mają mniejsze głębokości, więc są łatwiejsze w opisie i szybsze w klasyfikacji. 
W tabelach \ref{tab:dep_wav} oraz \ref{tab:dep_dig} widzimy jak wyglądały średnie głębokości dla drzew wygenerowanych przy kroswalidacji.

\begin{table}
\caption{Średnie głębokości drzew wytrenowanych dla zbioru danych WAVEFORM}\label{tab:dep_wav}
 \begin{tabular}{l || P{2.5cm} | P{2.5cm} | P{2.5cm}}
  Metoda/parametr     & Drzewo decyzyjne & Drzewo decyzyjne SVM & Drzewo decyzyjne genetycznie \\ 
  \hline
  średnia głębokość drzewa & 6.4     & 3.0     & 2.4  \\
 \end{tabular}
\end{table}

\begin{table}
\caption{Średnie głębokości drzew wytrenowanych dla zbioru danych digits}\label{tab:dep_dig}
 \begin{tabular}{l || P{2.5cm} | P{2.5cm} | P{2.5cm}}
  Metoda/parametr     & Drzewo decyzyjne & Drzewo decyzyjne SVM & Drzewo decyzyjne genetycznie \\ 
  \hline
  średnia głębokość drzewa & 9.4     & 4.0     & 3.7  \\
 \end{tabular}
\end{table}

Drzewa decyzyjne wygenerowane algorytmem genetycznym oraz budowane z nich lasy losowe pozwalają osiągnąć najlepszą skuteczność przy jednoczesnej największej prostocie
drzewa. Odbywa się to jednak kosztem złożności czasowej, jednak dzięki użyciu Sparka czas generowania drzewa znacząco spada. 
Również sterując parametrami stopu (przycinania), możemy decydować o kompromisie pomiędzy niedouczeniem/mniejszym czasem wykonania a przeuczeniem. 

\section{Podsumowanie i możliwe dalsze prace}
Wyniki uzyskiwane metodą generowania hiperpłaszczyzn przez algorytm genetyczny są dość obiecujące. 
Na testowanych zbiorach danych osiagały znacznie lepsze rezultaty od baseline'u, którym była zwykła metoda drzew decyzyjnych.
Wyniki eksperymentów związanych ze skalowalnością pokazują, że używanie do obliczeń więcej niż 10 jednostek obliczeniowych jest
nieopłacalne, prawdopodobnie ze względy na straty związane z przesyłaniem danych po sieci. Dlatego należałoby uruchomić więcej
workerów w ramach jednej jednostki. Niestety sprzęt w laboratorium MIMUW uniemożliwiał przeprowadzenie poważniejszych obliczeń,
jak i testowanie na większej liczbie danych.
Możliwe dalsze kierunki rozwoju mogłyby pójśc w stronę optymalizacji biblioteki (pisanie niektórych fragmentów kodu w C)
oraz podpięcia pod GPU. Prace teoretyczne mogłyby pójśc w stronę badania różnych strategii momentu stopu oraz radzenia sobie 
z bliskimi punktami, co opisywałem w rozdziale \ref{sec:strategy}.

\newpage
\appendix
\section{\\Kod algorytmu szukania optymalnych hiperplaszczyzn} \label{App:hyperp}

\begin{lstlisting}
# dla wszyskich funkcji:

# dec <- lista decyzji
# table <- tabela 
# attrs_list <- lista atrybutow

# ekstracja nowych cech, generowanie nowej tabeli
def extract_features(sc=None):
        extracted_table = []
        while True:
            unconsistent_groups = ConsistentChecker.
				  count_unconsistent_groups(
				  extracted_table, dec, sc)
# kompresja niespojnych obszarow, np wyrzucanie 
# bliskch punktow o roznych decyzjach
            unconsistent_groups = filter(
				  lambda x: 
				  dpoints_strategy.decision(x) is None, 
				  unconsistent_groups)
# jesli sa jeszcze jakies niespojne grupy szukamy hiperplaszczyzn
            if unconsistent_groups:
            
                best_hyperplane = search_best_hyperplane(
				  unconsistent_groups, sc)
                extracted_table.append(
				count_objects_positions(best_hyperplane))
            else:
                break

        return np.transpose(np.array(extracted_table))

# szukanie optymalnej hiperplaszczyzny
def search_best_hyperplane(unconsistent_groups, sc=None):

            rdd_attributes = sc.parallelize(attrs_list)
# zrownoleglamy ze wzgledu na osie (atrybuty)
# i liczymy optymalne hiperplaszczyzny dla rzutow
            hyperplanes = rdd_attributes.map(
			  lambda x: 
			  search_best_hyperplane_for_projection(
			  x, unconsistent_groups)).collect()
	
# zwracamy najlepsza
        return max(hyperplanes, key=lambda x: x[1][0])

# szukanie optymalnej hiperplaszczyzny dla rzutu
def search_best_hyperplane_for_projection(
    attr, unconsistent_groups, sc=None):

        projection_axis = table[attr]
        other_axes = [x for x in attrs_list if not x == attr]
        new_table = table[other_axes]
	
	# szukamy optymalnej hiperplaszczyzny algorytmem genetycznym
        gen_search = GeneticSearch(len(other_axes), dec, new_table,
                                   projection_axis, unconsistent_groups)
        cand_hyperplane = gen_search.genetic_search(sc)

        return attr, cand_hyperplane

# unconsistent_groups <- lista niespojnych grup
# population_size <- rozmiar populacji
# przeszukiwanie genetyczne
def genetic_search(sc=None):

# inicjowanie populacji
        population = init_generation()

        current_best_award = 0
        the_same_awards = 0
        for i in range(max_iter):
# liczymy funkcje celu dla populacji
            rdd_population = sc.parallelize(population, 
					    population_size * 10)
            awards = rdd_population.mapPartitions(
				    count_award_for_chunk).collect()
	    
# wybieramy najlepszych osobników
            population_awards = select_best_individuals(awards)

            best_individual = population_awards[0]

            population = map(lambda x: x[1], population_awards)

            new_generation = count_new_generation(
			     copy.deepcopy(population))

            population = new_generation + population

        return best_individual
        
def count_award_for_chunk(self, population):
        for individual in population:
            yield self.count_award(individual)
            
def count_award(individual):
        all_objects = reduce(add, unconsistent_groups)
# liczymy rzuty wszystkich obiektow
        projections = count_projections(individual, all_objects)
# sortujemy po wartosci rzutu
        objects = sorted(projections, key=lambda x: x[1])

# liczenie funkcji oceny (discMeasure) dla posortowanych obiektow
        object_group_dict = {}
        group_fqs_dict = {}
        act_left_sum = {}
        act_right_sum = {}

        for ind, group in enumerate(unconsistent_groups):
            group_decisions = []
            for obj in group:
                group_decisions.append(self.dec[obj])
                object_group_dict[obj] = ind
            act_left_sum[ind] = 0
            act_right_sum[ind] = len(group_decisions)
            group_fqs_dict[ind] = DiscMeasureCalculator.
				 prepare_hist(group_decisions)

        act_award = 0
        max_award = 0
        for obj, proj in objects:
            group_id = object_group_dict[obj]
            dec = self.dec[obj]
            act_left_sum[group_id], act_right_sum[group_id], act_award = \
                DiscMeasureCalculator.
	        update_award(dec, act_left_sum[group_id], 
			     act_right_sum[group_id],
                             group_fqs_dict[group_id], act_award)
            if act_award > max_award:
                max_award = act_award
                good_proj = proj
                
# zwracamy wartosc najlepszego wyniku, osobnika (wektory) i przesuniecie
        return max_award, individual, good_proj  

# wyliczanie rzutu dla osobnika i grupy obiektow 
def count_projections(individual, objects):
        table = table_as_matrix[objects, :]
        proj = [projection_axis[obj] for obj in objects]
        return zip(objects, proj - table.dot(individual))
\end{lstlisting}


\begin{thebibliography}{9}

\bibitem{Rudiments}
Z. Pawlak, A. Skowron,
\emph{Rudiments of rough sets},
Information Sciences, {\bf 177}(1) (2007) 3--27.

\bibitem{Discretization}
  Hun Son Nguyen,
  \emph{Approximate Boolean Reasoning: Foundations and Applications in Data Mining},
  Rozprawa Doktorska, Instytut Matematyki, Uniwersytet Warszawski Banacha 2, 02-097, Warszawa, Polska
  1998.
\bibitem{Hyperplanes}
  Hun Son Nguyen,
  \emph{From Optimal Hyperplanes to Optimal Decision Trees}.
  Instytut Matematyki, Uniwersytet Warszawski Banacha 2, 02-097, Warszawa, Polska
  1998.
\bibitem{Genetic}
  Z. Michalewicz, 
  \emph{Algorytmy genetyczne + struktury danych = programy ewolucyjne}.
  Warszawa, WNT, 
  1996. (pol.)
\bibitem{BooleanGenetic}
  Srilatha Chebrolu, Sriram G Sanjeevi,
  \emph{Rough set theory for discretization based on boolean reasoning and 
  genetic algorithm}.
  Department of Computer Science and Engineering, NIT Warangal, India,
  2012.
\bibitem{RandomForest}
  Tin Kam Ho,
  \emph{Random Decision Forests}.
  AT\&T Bell Laboratories 600 Mountain Avenue, 2C-548C Murray Hill, NJ 07974, USA
\bibitem{SVM}
  C. Cortes, V. Vapnik
  \emph{Support-vector networks}
  Machine Learning 20 (3): 273–297. (1995)
\bibitem{Scalability}
  Xiaodong Zhang Xiaodong Zhang, Yong Yan Yong Yan, Qian Ma Qian Ma
  \emph{Measuring and Analyzing Parallel Computing Scalability}
  Parallel Processing, 1994. ICPP 1994 Volume 2. International Conference
\bibitem{MapReduce}
Jeffrey Dean and Sanjay Ghemawat
\emph{MapReduce: Simplified Data Processing on Large Clusters}.
Google Inc, 2004
\bibitem{SparkDocumentation}
 Spark Documentation,
 \emph{http://spark.apache.org/docs/latest/programming-guide.html}
\bibitem{SparkDocumentation}
 Python Documentation,
 \emph{https://www.python.org/doc}
\bibitem{SklearnDocumentation}
 Sklearn Documentation,
 \emph{http://scikit-learn.org/stable/}
 
\end{thebibliography}
 
\end{document}